
\documentclass{article}
\usepackage{amsmath}
\usepackage{amssymb}
\usepackage{array}
\usepackage{booktabs}
\usepackage{listings}
\usepackage{color}
\usepackage{graphicx}
\usepackage{tikz}

\usepackage[paperwidth=8.5in,paperheight=11.0in,
  left=0.35in,right=0.35in,top=0.3in,bottom=0.15in,
  includefoot,heightrounded]{geometry}

\usepackage{fancyhdr}
\pagestyle{fancy}
\fancyhf{} % Clear existing headers/footers
\fancyhead[R]{\textbf{Shah958, Jaanav Shah}} % Right-aligned header
\renewcommand{\headrulewidth}{0pt} % Remove header rule (optional)
\fancyfoot[R]{\textbf{Shah958, Jaanav Shah} \quad | \quad \textbf{Page \thepage}} % Right-aligned footer with page number
\renewcommand{\footrulewidth}{0pt} % Remove footer rule (optional)
\definecolor{codegreen}{rgb}{0,0.6,0}
\definecolor{codegray}{rgb}{0.5,0.5,0.5}
\definecolor{codepurple}{rgb}{0.58,0,0.82}
\definecolor{backcolour}{rgb}{0.95,0.95,0.92}

\lstdefinestyle{mystyle}{
    backgroundcolor=\color{backcolour},   
    commentstyle=\color{codegreen},
    keywordstyle=\color{magenta},
    numberstyle=\tiny\color{codegray},
    stringstyle=\color{codepurple},
    basicstyle=\footnotesize,
    breakatwhitespace=false,         
    breaklines=true,                 
    captionpos=b,                    
    keepspaces=true,                 
    numbers=left,                    
    numbersep=5pt,                  
    showspaces=false,                
    showstringspaces=false,
    showtabs=false,                  
    tabsize=2
}

\lstset{style=mystyle}

\begin{document}


\begin{center}
    {\Huge \textbf{ECON 34000 Midterm 1 Study Guide}}
\end{center}

\section*{Section 1a}

\section*{1. Preferences and Rationality}

\subsection*{ (a) Basic Definitions}

Economists assume that individual behavior adheres to the \textbf{optimization principle} : a decision-maker chooses their most preferred option from the set of available alternatives. To analyze this, we define preferences over consumption alternatives.

\begin{center}
\begin{tabular}{>{\bfseries}l l p{11cm} l}
\toprule
\textbf{Concept} & \textbf{Notation} & \textbf{Definition and Interpretation} & \textbf{Source} \\
\midrule
Alternative/Bundle & $x, y$ & A complete list of goods/services being chosen. & \\
Weak Preference & $x \succeq y$ & The decision-maker (weakly) prefers $x$ over $y$, meaning $x$ is liked at least as much as $y$. & \\
Strict Preference & $x \succ y$ & $x$ is strictly preferred to $y$. Formally: $x \succ y \iff x \succeq y$ but not $y \succeq x$. & \\
Indifference & $x \sim y$ & The decision-maker is indifferent between $x$ and $y$. Formally: $x \sim y \iff x \succeq y$ and $y \succeq x$. & \\
\bottomrule
\end{tabular}
\end{center}

\subsection*{Rational Preferences (Axioms of Rationality)}

A preference relation ($\succeq$) is considered \textbf{rational} in the economic sense if it satisfies two fundamental consistency axioms: Completeness and Transitivity.

\begin{enumerate}
    \item \textbf{Completeness}
    \begin{itemize}
        \item \textbf{Definition:} The consumer can always compare any two alternatives.
        \item \textbf{Formal Statement (LaTeX):} \[\text{For any bundles } x \text{ and } y \text{, either } x \succeq y \text{ or } y \succeq x \text{ (or both).}\]
        \item \textbf{Interpretation:} The consumer is always able to make a choice between any two given bundles.
    \end{itemize}
    \item \textbf{Transitivity}
    \begin{itemize}
        \item \textbf{Definition:} Preferences must be logically consistent across multiple alternatives.
        \item \textbf{Formal Statement (LaTeX):} \[\text{If } x \succeq y \text{ and } y \succeq z \text{, then } x \succeq z \text{ for any bundles } x, y, z.\]
        \item \textbf{Interpretation:} If preferences were not transitive, there could be a set of bundles for which there is no best choice, undermining the maximization premise.
    \end{itemize}
\end{enumerate}

\noindent\rule{\textwidth}{0.4pt}

\subsection*{$\star$ Exam Focus: Shortcuts \& Mistakes}

\begin{itemize}
    \item \textbf{Pitfall: Intransitivity = Exploitation:} Violations of transitivity ($x \succ y \succ z \succ x$) allow for exploitation, often demonstrated by the "money pump" scenario, where an individual can be repeatedly convinced to trade away wealth simply to cyclically regain their preferred position.
    \item \textbf{Key Shortcut: Monotonicity Implies Strict Preference:} When evaluating preferences, especially in the context of convexity, remember the Monotonicity assumption (often implied by "well-behaved" preferences): if bundle $y$ contains strictly more of at least one good than $x$, $y \succ x$.
\end{itemize}

\noindent\rule{\textwidth}{0.4pt}

\subsection*{Exam-Style Problem 1: Testing Rationality and Convexity}

\textbf{Problem (Adapted from PS1 Q3):}

Consider an individual's preferences over bundles $(x_1, x_2)$, where preferences are complete and transitive. You are given the following information: (i) The individual is indifferent between $X = (2, 6)$ and $Y = (8, 2)$. (ii) The preference is strictly monotonic (more is always preferred). (iii) The individual strictly prefers $X = (2, 6)$ over $Z = (5, 5)$.

Is this individual's preference convex? Explain your answer precisely using the definitions of the preference axioms.

\textbf{Solution:}

\textbf{Goal:} Test if the mixture of two indifferent bundles is weakly preferred to the extremes, as required by convexity.

\begin{enumerate}
    \item \textbf{Define the Mixture Bundle ($M$):} The midpoint (50-50 mixture) of $X=(2, 6)$ and $Y=(8, 2)$ is $M$: \[M = \frac{1}{2}(2, 6) + \frac{1}{2}(8, 2) = (1+4, 3+1) = (5, 4)\]
    \item \textbf{Apply Monotonicity to Compare $M$ and $Z$:} We compare $Z=(5, 5)$ and $M=(5, 4)$. Since $Z$ has the same amount of good 1 and strictly more of good 2 than $M$, by strict monotonicity: \[Z \succ M\] (The bundle $(5, 5)$ is strictly preferred to $(5, 4)$).
    \item \textbf{Apply Transitivity to Compare $X$ and $M$:} We are given $X \succ Z$ (i.e., $(2, 6) \succ (5, 5)$) from condition (iii). From step 2, we found $Z \succ M$. By the Transitivity axiom: \[\text{If } X \succ Z \text{ and } Z \succ M \text{, then } X \succ M.\] \[(2, 6) \succ (5, 4)\]
    \item \textbf{Conclusion on Convexity:} For preferences to be convex, since $X \sim Y$, the mixture $M$ must satisfy $M \succeq X$ and $M \succeq Y$. However, we found that the mixture $M=(5, 4)$ is strictly inferior to $X=(2, 6)$ because $X \succ M$.
\end{enumerate}

\textbf{Answer:} The preference is \textbf{not convex}. It violates convexity because the mixture bundle $M=(5, 4)$ is strictly inferior to the original bundles $X=(2, 6)$ and $Y=(8, 2)$ (since $X \sim Y$ and $X \succ M$). The individual prefers the extremes over the mixture, indicating concave preferences.

\hrulefill

\section{Preferences and Rationality (Continued)}

\subsection{Representing Preferences}

Preferences are typically visualized using \textbf{Indifference Curves (ICs)}. An IC represents all consumption bundles that yield the same level of satisfaction for the consumer.

\subsubsection*{Definitions and Properties}

\begin{itemize}
    \item \textbf{Indifference Curve (IC):} The set of all consumption bundles $y = (y_1, y_2)$ that are equally attractive (indifferent) to a specific bundle $x = (x_1, x_2)$. \[\text{I}(x) = \{y \in \mathbb{R}^2_+ \mid y \sim x\}\]

    \item \textbf{Weakly Preferred Set:} The collection of all bundles weakly preferred to $x$. This is the area on and above the indifference curve $I(x)$.

    \item \textbf{Increasing Preference:} Higher indifference curves (further from the origin) correspond to higher levels of utility/preference.
\end{itemize}

\begin{center}
    \begin{tabular}{llll}
        \toprule
        Key Property & Condition & Implication for IC Shape & Source \\
        \midrule
        \textbf{Rationality} & Complete and Transitive & ICs cannot cross & \\
        \textbf{Monotonicity} & More is always better & ICs must be downward sloping & \\
        \textbf{Convexity} & Averages are preferred to extremes & ICs must be bowed toward the origin; MRS must be diminishing & \\
        \bottomrule
    \end{tabular}
\end{center}

$\star$ \textbf{Exam Focus: Shortcuts \& Mistakes}

\begin{itemize}
    \item \textbf{Common Mistake: Crossing ICs:} If two distinct ICs (representing different utility levels) cross, it violates the \textbf{Transitivity} axiom.

    \item \textbf{Shortcut: Slope Check:} If preferences are monotonic (which is usually assumed for "goods"), you only need to check convexity (diminishing MRS). Concave preferences (e.g., $u(x_1, x_2) = x_1^2 + x_2^2$) imply choosing corner solutions.
\end{itemize}

\subsubsection*{Marginal Rate of Substitution (MRS)}

The MRS is the instantaneous slope of the indifference curve at a specific point.

\begin{itemize}
    \item \textbf{Definition:} The rate at which the consumer is just willing to substitute good 2 ($\Delta x_2$) for a marginal change in good 1 ($\Delta x_1$), holding utility constant.

    \item \textbf{Formula (LaTeX):} The MRS is the negative ratio of the marginal utilities (MU): \[\text{MRS}(x_1, x_2) = \frac{dx_2}{dx_1} \bigg\rvert_{u(x_1, x_2)=constant} = -\frac{\text{MU}_1}{\text{MU}_2} = -\frac{\partial u(x_1, x_2)/\partial x_1}{\partial u(x_1, x_2)/\partial x_2}\]
\end{itemize}

\subsubsection*{Derivation of MRS (Calculus Method)}

We hold utility constant along the curve, meaning the total change in utility ($du$) is zero: \[du = \frac{\partial u}{\partial x_1} dx_1 + \frac{\partial u}{\partial x_2} dx_2 = 0\] Rearranging the total differential to find the slope $\frac{dx_2}{dx_1}$: \[\frac{\partial u}{\partial x_2} dx_2 = -\frac{\partial u}{\partial x_1} dx_1\] \[\text{MRS} = \frac{dx_2}{dx_1} = -\frac{\partial u/\partial x_1}{\partial u/\partial x_2}\]

\subsubsection*{Testing for Convexity (Diminishing MRS)}

A preference relation is \textbf{convex} if the MRS is diminishing.

\begin{itemize}
    \item \textbf{Mathematical Test:} To verify convexity, check if the absolute value of the MRS, $|\text{MRS}|$, decreases as $x_1$ increases along the indifference curve. \[\text{Convexity } \iff \frac{d|\text{MRS}|}{dx_1} < 0\] (Alternatively, since $\text{MRS}$ is typically negative for goods, check $\frac{d(\text{MRS})}{dx_1} > 0$).
\end{itemize}

\vspace{1em}\hrule\vspace{1em}

\subsubsection*{Step-by-Step Worked Example: Concave Preferences}

Determine if the utility function $u(x_1, x_2) = x_1 + x_2^2$ represents convex preferences by analyzing its MRS.

\textbf{Step 1: Calculate Marginal Utilities ($MU$)} \[MU_1 = \frac{\partial u}{\partial x_1} = 1\] \[MU_2 = \frac{\partial u}{\partial x_2} = 2x_2\]

\textbf{Step 2: Calculate the Marginal Rate of Substitution ($MRS$)} \[\text{MRS} = -\frac{MU_1}{MU_2} = -\frac{1}{2x_2}\]

\textbf{Step 3: Express $x_2$ in terms of $x_1$ along an Indifference Curve ($u(x_1, x_2) = k$)} \[k = x_1 + x_2^2 \implies x_2 = \sqrt{k - x_1}\]

\textbf{Step 4: Analyze Diminishing MRS by checking $\frac{d(\text{MRS})}{dx_1}$}
Since $\text{MRS} = -\frac{1}{2x_2}$, we need to see how $x_2$ changes as $x_1$ increases, keeping $k$ constant. As $x_1$ increases, $x_2 = \sqrt{k - x_1}$ must decrease (since $k$ is constant). Since $x_2$ decreases, the denominator $2x_2$ decreases, making the magnitude of the $\text{MRS}$ (i.e., $\frac{1}{2x_2}$) \textbf{increase}.

\textit{Alternatively, formally using the second derivative test, recall the condition for convexity is $\frac{d^2 x_2}{dx_1^2} > 0$. The solution for this specific function yields:} \[\frac{d^2 x_2}{dx_1^2} = -\frac{k}{(k - x_1^2) \sqrt{k - x_1^2}} < 0\]

\textbf{Conclusion:} Since the second derivative is negative, the MRS is increasing (not diminishing). This indicates \textbf{concave preferences} , and thus the preferences are \textbf{not convex}.

\vspace{1em}\hrule\vspace{1em}

\subsubsection*{Exam-Style Problem 2: Indifference Curve Consistency}

\textbf{Problem (Based on Varian Review Questions**):}

Suppose a consumer has preferences over two goods, $X$ and $Y$, that are strictly monotonic and rational (complete and transitive). Prove why two distinct indifference curves, $I_1$ and $I_2$, representing different levels of preference, cannot intersect.

\textbf{Solution:}

We use proof by contradiction, based on the axioms of \textbf{Rationality} (Transitivity) and \textbf{Monotonicity}.

\begin{enumerate}
    \item \textbf{Assumption (Contradiction Setup):} Assume two distinct indifference curves, $I_1$ and $I_2$, intersect at point $Z$.

    \item \textbf{Define Points:}
    \begin{itemize}
        \item Let $X$ be a bundle on $I_1$ that is not $Z$.
        \item Let $Y$ be a bundle on $I_2$ that is not $Z$.
        \item Since $I_1$ and $I_2$ are distinct, one curve must represent strictly higher preference. Assume bundles on $I_1$ are strictly preferred to bundles on $I_2$, so $X \succ Y$.
    \end{itemize}

    \item \textbf{Apply Indifference:}
    \begin{itemize}
        \item Since $X$ and $Z$ are on the same curve $I_1$: $X \sim Z$.
        \item Since $Y$ and $Z$ are on the same curve $I_2$: $Y \sim Z$.
    \end{itemize}

    \item \textbf{Apply Transitivity:}
    \begin{itemize}
        \item If $X \sim Z$ and $Y \sim Z$, then by Transitivity ($X \succeq Z$ and $Z \succeq Y$ implies $X \succeq Y$) and Indifference, we must conclude that $X \sim Y$.
    \end{itemize}

    \item \textbf{Contradiction:} Step 4 states $X \sim Y$, but Step 2 assumes $X \succ Y$. This is a contradiction.

    \item \textbf{Conclusion:} The initial assumption that two distinct indifference curves can intersect must be false. (Monotonicity further ensures that ICs are downward sloping, preventing a single IC from crossing itself).
\end{enumerate}

\hrule

\subsection*{Key Results: Utility Representation and Ordinality}

The primary function of preference axioms (Completeness and Transitivity) is to ensure that preferences can be mathematically represented, which allows for calculation and optimization.

\textbf{Utility Representation}
\begin{itemize}
    \item \textbf{Definition: Utility Function ($u$):} A function that assigns a real number (utility level) to every consumption bundle such that preferred bundles receive higher numbers.
    \begin{itemize}
        \item Formally, $u(x)$ represents the preference relation $\succeq$ if and only if: \begin{align*} x \succ y &\iff u(x) > u(y) \\ x \sim y &\iff u(x) = u(y) \end{align*}
    \end{itemize}
    \item \textbf{Existence Condition:} A preference relation can be represented by a utility function if it is \textbf{rational} (Complete and Transitive) and satisfies a technical \textbf{continuity} assumption (small changes in a bundle result in small changes in preference level).
\end{itemize}

\textbf{Utility is Ordinal, Not Cardinal}

The specific numerical value generated by a utility function (e.g., $u(x)=10$) has no intrinsic meaning; only the ranking it imposes matters. Utility measures relative preference, not absolute happiness or intensity.

\begin{itemize}
    \item \textbf{Ordinal Utility:} Utility is \textbf{ordinal} , meaning it only reflects the order or ranking of bundles. Saying $u(x)=4$ and $u(y)=2$ means $x$ is preferred to $y$, not that $x$ is liked "twice as much" as $y$.
    \item \textbf{Monotonic Transformations:} Since only the ranking matters, any strictly increasing (monotonic) transformation of a utility function represents the exact same preferences.
    \begin{itemize}
        \item \textbf{Definition: Monotonic Transformation ($f$):} A function $f(u)$ such that for any $u_A, u_B$: \[u_A > u_B \iff f(u_A) > f(u_B)\]
        \item \textbf{Examples of Valid Monotonic Transformations:}
        \begin{itemize}
            \item Adding a constant: $v(x) = u(x) + c$
            \item Multiplying by a positive constant: $v(x) = A \cdot u(x)$, where $A > 0$
            \item Taking the logarithm (if $u > 0$): $v(x) = \ln(u(x))$
        \end{itemize}
    \end{itemize}
\end{itemize}

\noindent\rule{\linewidth}{0.4pt}

\noindent $\star$ \textbf{Exam Focus: Shortcuts \& Mistakes}

\begin{itemize}
    \item \textbf{Shortcut: Testing Equivalent Preferences:} Two utility functions, $u$ and $v$, represent the same preferences if and only if they result in the \textbf{same Marginal Rate of Substitution (MRS)} as a function of $x_1$ and $x_2$. \[\text{If } \text{MRS}_u(x_1, x_2) = \text{MRS}_v(x_1, x_2) \text{, then } u \text{ and } v \text{ are equivalent.}\] (This holds because MRS defines the slope of the indifference curves, and if the curves have the same slope everywhere, they define the same shape of indifference map).
    \item \textbf{Common Mistake: Invalid Transformation:} A common mistake is using a non-monotonic transformation. For example, $f(u)=u^2$ is only a monotonic transformation if $u$ is strictly positive. If $u$ could be negative, squaring it would reverse the ranking of negative utility bundles [174, Answer A13].
\end{itemize}

\noindent\rule{\linewidth}{0.4pt}

\noindent\textbf{Step-by-Step Worked Example: Testing Equivalent Preferences}

Determine if the utility functions $u(x_1, x_2) = x_1^2 x_2^2$ and $v(x_1, x_2) = \ln x_1 + \ln x_2$ represent the same preferences.

\textbf{Strategy:} Compare the MRS derived from each function.

\textbf{Step 1: Calculate $MRS_u$ for $u(x_1, x_2) = x_1^2 x_2^2$} \[MU_1^u = \frac{\partial u}{\partial x_1} = 2x_1 x_2^2\] \[MU_2^u = \frac{\partial u}{\partial x_2} = 2x_1^2 x_2\] \[MRS_u = -\frac{MU_1^u}{MU_2^u} = -\frac{2x_1 x_2^2}{2x_1^2 x_2} = -\frac{x_2}{x_1}\]

\textbf{Step 2: Calculate $MRS_v$ for $v(x_1, x_2) = \ln x_1 + \ln x_2$} \[MU_1^v = \frac{\partial v}{\partial x_1} = \frac{1}{x_1}\] \[MU_2^v = \frac{\partial v}{\partial x_2} = \frac{1}{x_2}\] \[MRS_v = -\frac{MU_1^v}{MU_2^v} = -\frac{1/x_1}{1/x_2} = -\frac{x_2}{x_1}\]

\textbf{Step 3: Conclusion}
Since $MRS_u = MRS_v = -x_2/x_1$, the two utility functions define the exact same set of indifference curves and thus represent the same preferences.

\textit{Self-Check: $v(x_1, x_2)$ is a monotonic transformation of $u(x_1, x_2)$ if we take the fourth root of $u$ and then the logarithm: $f(u) = \ln(\sqrt{u}) = \ln(x_1^{1/2} x_2^{1/2}) = \frac{1}{2}\ln x_1 + \frac{1}{2}\ln x_2$. Since $\ln x_1 + \ln x_2$ is just $2 \cdot (\frac{1}{2}\ln x_1 + \frac{1}{2}\ln x_2)$, they are indeed equivalent up to a scale factor.}

\noindent\rule{\linewidth}{0.4pt}

\noindent\textbf{Exam-Style Problem 3: Preferences that Differ}

\textbf{Problem (Based on PS1 Q4):}

Argue why the utility functions $u(x_1, x_2) = x_1 + x_2$ and $v(x_1, x_2) = 2x_1 + 3x_2$ do \textbf{not} represent the same preferences.

\textbf{Solution:}

We can prove that two utility functions do not represent the same preferences if we can find at least one pair of bundles ($X$, $Y$) that are ranked differently by the two functions.

\begin{enumerate}
    \item \textbf{Define Test Bundles:} Let $X = (1, 0)$ and $Y = (0, 1)$.
    \item \textbf{Evaluate Ranking using $u(x_1, x_2) = x_1 + x_2$:} \[u(X) = 1 + 0 = 1\] \[u(Y) = 0 + 1 = 1\] According to $u$, the consumer is indifferent: $X \sim Y$.
    \item \textbf{Evaluate Ranking using $v(x_1, x_2) = 2x_1 + 3x_2$:} \[v(X) = 2(1) + 3(0) = 2\] \[v(Y) = 2(0) + 3(1) = 3\] According to $v$, the consumer strictly prefers $Y$ over $X$: $Y \succ X$ since $3 > 2$.
    \item \textbf{Conclusion:} Since the two utility functions rank the bundles $X$ and $Y$ differently ($X \sim Y$ under $u$, but $Y \succ X$ under $v$), they do \textbf{not} represent the same underlying preferences. \textit{(Alternatively, note that $\text{MRS}_u = -1$ and $\text{MRS}_v = -2/3$. Since the MRS values differ, the slope of their respective indifference curves differs, meaning the preferences cannot be the same).}
\end{enumerate}

\hrulefill

\section{2. Commodities and Utility}
\subsection{(a) Commodities}

In applied microeconomics, the consumption space is typically defined over \textbf{commodities} or products. We assume consumers can choose arbitrary positive units of each commodity.

\begin{itemize}
    \item \textbf{Commodity Bundle (Vector Notation):} A bundle $x$ is a vector listing the chosen quantities of $N$ products.
    $$\text{Bundle: } x = (x_1, x_2, \dots, x_N) \in \mathbb{R}^N_+$$
    For graphical analysis in ECON 340, we typically focus on $N=2$, meaning a bundle $x=(x_1, x_2) \in \mathbb{R}^2_+$.
\end{itemize}

\subsubsection*{Classification of Commodities}

Commodities are categorized based on whether the consumer prefers more, less, or is indifferent to changes in quantity.

\begin{enumerate}
    \item \textbf{Good:} A commodity of which the consumer prefers to consume \textbf{more}.
    \begin{itemize}
        \item[\circ] \textit{Example:} Leisure.
    \end{itemize}
    \item \textbf{Bad:} A commodity of which the consumer prefers to consume \textbf{less}.
    \begin{itemize}
        \item[\circ] \textit{Example:} Work, pollution, or too much of a desirable good (satiation/bliss point).
    \end{itemize}
    \item \textbf{Neutral:} A commodity where the consumer is \textbf{indifferent} between consuming more or less.
\end{enumerate}

\noindent $\star$ \textbf{Exam Focus: Shortcuts \& Mistakes}

\begin{itemize}
    \item \textbf{Relationship to Monotonicity:} If preferences are \textbf{strictly monotone} (a common assumption for "well-behaved" preferences), then \textit{all} commodities are strictly goods, meaning $y \succ x$ if $y$ has strictly more of at least one good and at least as much of all others.
    \item \textbf{Indifference Curve Slope:} The standard negative slope assumption only holds if both commodities are \textit{goods} (or both are \textit{bads}). If one is a good and one is a bad, the IC slopes upward. If one is a good and one is a neutral, the IC is horizontal or vertical (see example below).
\end{itemize}

\noindent\rule{\linewidth}{0.4pt}

\subsubsection*{Step-by-Step Worked Example: Good and Neutral Preferences}

Determine the shape of the indifference curves and direction of increasing preference if the consumer is choosing between Good 1 ($x_1$, a \textit{Good}) and Good 2 ($x_2$, a \textit{Neutral}).

\paragraph*{Step 1: Define the Utility Relationship}
Since $x_1$ is a Good, increasing $x_1$ increases utility. Since $x_2$ is Neutral, changing $x_2$ does not affect utility. Utility depends only on the quantity of Good 1:
$$u(x_1, x_2) = v(x_1)$$

\paragraph*{Step 2: Determine the Marginal Utilities ($MU$) and MRS}
$$MU_1 = v'(x_1) > 0$$
$$MU_2 = \frac{\partial u}{\partial x_2} = 0$$
The Marginal Rate of Substitution (MRS) formula requires $MU_2 \neq 0$, but conceptually, the slope of the IC is the rate of trade that holds utility constant. To keep $u$ constant, $x_1$ must remain constant:
$$du = MU_1 dx_1 + MU_2 dx_2 = 0$$
$$MU_1 dx_1 + 0 \cdot dx_2 = 0$$
Since $MU_1 > 0$, we must have $dx_1 = 0$ along any indifference curve.

\paragraph*{Step 3: Sketch Indifference Curves and Preference Direction}
Since $x_1$ must be constant along the indifference curve, the curve is a vertical line. Since $x_1$ is a Good, increasing $x_1$ increases preference.

\begin{itemize}
    \item \textbf{IC Shape:} Vertical lines.
    \item \textbf{Preference Direction:} To the right (increasing $x_1$).
\end{itemize}

\noindent\rule{\linewidth}{0.4pt}

\subsubsection*{Exam-Style Problem 4: Defining Commodity Types by Ranking}

\paragraph*{Problem (Conceptual Check):}
A consumer has preferences over two items: pizza slices ($x_1$) and spam emails ($x_2$). She is indifferent between Bundle $A=(5, 10)$ and Bundle $B=(5, 15)$. She strictly prefers Bundle $C=(6, 12)$ over Bundle $A=(5, 10)$.

Based solely on these two rankings and the strict preference axiom, classify $x_1$ and $x_2$ as a Good, a Bad, or a Neutral.

\paragraph*{Solution:}
\begin{enumerate}
    \item \textbf{Analyze $x_2$ using Bundles $A$ and $B$:}
    \begin{itemize}
        \item[\circ] $A = (5, 10)$ and $B = (5, 15)$. $B$ has the same amount of $x_1$ but strictly more $x_2$ ($15 > 10$).
        \item[\circ] The consumer is indifferent: $A \sim B$.
        \item[\circ] \textbf{Conclusion:} Since increasing $x_2$ while holding $x_1$ constant does not change the preference ranking, $x_2$ is a \textbf{Neutral} commodity.
    \end{itemize}
    \item \textbf{Analyze $x_1$ using Bundles $A$ and $C$:}
    \begin{itemize}
        \item[\circ] $A = (5, 10)$ and $C = (6, 12)$. $C$ has strictly more $x_1$ ($6 > 5$) and more $x_2$ ($12 > 10$).
        \item[\circ] We are given $C \succ A$.
        \item[\circ] To isolate the effect of $x_1$, we use the conclusion that $x_2$ is Neutral (Step 1). Since $A \sim B$, if the consumer prefers $C$ to $A$, she must also prefer $C$ to $B$ (by transitivity, since $C \succ A \sim B$ implies $C \succ B$).
        \item[\circ] Consider Bundle $D = (6, 10)$. By the definition of $x_2$ (Neutral), $C = (6, 12) \sim D = (6, 10)$ because changing $x_2$ doesn't matter.
        \item[\circ] We know $C \succ A$, and $C \sim D$, so $D \succ A$ (by transitivity).
        \item[\circ] $D=(6, 10)$ has strictly more $x_1$ than $A=(5, 10)$.
        \item[\circ] \textbf{Conclusion:} Since increasing $x_1$ while holding $x_2$ constant strictly increases the preference ranking, $x_1$ is a \textbf{Good} commodity. D$, so $D \succ A$ (by transitivity).
        \item[\circ] $D=(6, 10)$ has strictly more $x_1$ than $A=(5, 10)$.
        \item[\circ] \textbf{Conclusion:} Since increasing $x_1$ while holding $x_2$ constant strictly increases the preference ranking, $x_1$ is a \textbf{Good} commodity.
    \end{itemize}
\end{enumerate}

\noindent\rule{\linewidth}{0.4pt}

\section{Commodities and Utility (Continued)}
\subsection{Utility}

A utility function is the necessary mathematical tool that allows us to describe and analyze rational preferences.

\subsubsection*{Clear Definitions of Key Concepts}
\begin{itemize}
    \item \textbf{Utility Function ($u$):} A representation of a consumer's preferences ($\succeq$) that assigns a numerical value to every bundle such that more preferred bundles receive higher numbers. \begin{align*} x \succ y &\iff u(x) > u(y) \\ x \sim y &\iff u(x) = u(y) \end{align*}

    \item \textbf{Marginal Utility ($\text{MU}_i$):} The rate of change in utility ($u$) resulting from a marginal change in the consumption of good $i$, holding the consumption of all other goods constant. \[\text{MU}_i(x) = \frac{\partial u(x)}{\partial x_i}\] If preferences are strictly monotonic (more is better), then $\text{MU}_i > 0$.

    \item \textbf{Marginal Rate of Substitution (MRS):} The negative slope of the indifference curve at a point, representing the rate at which a consumer is willing to trade good 2 for good 1 while maintaining the same level of utility. \[\text{MRS}(x_1, x_2) = \frac{dx_2}{dx_1} \bigg\rvert_{u(x_1, x_2)=constant}\]
\end{itemize}

\subsubsection*{Relevant Formulas and Derivations (LaTeX)}

The MRS is formally defined as the negative ratio of marginal utilities: \[\text{MRS}_{1,2} = -\frac{\text{MU}_1}{\text{MU}_2} = -\frac{\partial u/\partial x_1}{\partial u/\partial x_2}\]

\textbf{Derivation of MRS (Total Differential Method):} We consider a small change $(dx_1, dx_2)$ along an indifference curve such that total utility change ($du$) is zero. \[du = \text{MU}_1 dx_1 + \text{MU}_2 dx_2 = 0\] Rearranging the equation to solve for the slope $\frac{dx_2}{dx_1}$: \[\text{MU}_2 dx_2 = -\text{MU}_1 dx_1\] \[\frac{dx_2}{dx_1} = -\frac{\text{MU}_1}{\text{MU}_2}\]

\noindent\rule{\linewidth}{0.4pt}

\subsubsection*{$\star$ Exam Focus: Shortcuts \& Mistakes}
\begin{itemize}
    \item \textbf{Pitfall: Marginal Utility Scale:} While utility itself is ordinal (non-unique), Marginal Utility's \textit{magnitude} is also arbitrary; if $u$ doubles, $\text{MU}_i$ doubles. We cannot compare MU across different people or even across the same person if the utility function has been scaled.

    \item \textbf{Key Shortcut: MRS Uniqueness:} The MRS is independent of the utility representation because monotonic transformations cancel out in the ratio. This is the key insight for checking if two functions represent the same preferences.

    \item \textbf{Interpretation Pitfall (MRS Sign):} MRS is mathematically negative for goods (since ICs slope downward). However, in problem solving, the absolute value ($|\text{MRS}|$) is often used to refer to the positive trade-off rate (e.g., "willing to give up 2 units of good 2").
\end{itemize}

\noindent\rule{\linewidth}{0.4pt}

\subsubsection*{Step-by-Step Worked Example: Calculating MRS for Quasilinear Preferences}

Find the MRS for the quasilinear utility function $u(x_1, x_2) = x_1 + \ln(x_2)$.

\textbf{Step 1: Calculate Marginal Utilities ($\text{MU}_i$)}
\[ \text{MU}_1 = \frac{\partial u}{\partial x_1} = 1 \]
\[ \text{MU}_2 = \frac{\partial u}{\partial x_2} = \frac{1}{x_2} \]

\textbf{Step 2: Calculate the Marginal Rate of Substitution (MRS)}
\[\text{MRS} = -\frac{\text{MU}_1}{\text{MU}_2} = -\frac{1}{1/x_2} = -x_2\]

\textbf{Step 3: Analyze Interpretation (Optional but recommended for exam clarity)}
$|\text{MRS}| = x_2$. The consumer's willingness to trade good 2 for good 1 depends only on the current amount of good 2 consumed, and not on the amount of good 1. If $x_2=5$, the consumer is willing to give up 5 units of good 2 for one marginal unit of good 1.

\noindent\rule{\linewidth}{0.4pt}

\subsubsection*{Exam-Style Problem 5: Preference Equivalence}

\textbf{Problem (Based on Varian Review Questions):}

A consumer's preferences are represented by the utility function $u(x_1, x_2) = \sqrt{x_1 x_2}$.
\begin{enumerate}
    \item Calculate the MRS for $u(x_1, x_2)$.
    \item Now consider the utility function $v(x_1, x_2) = x_1^2 x_2^2$. Does $v(x_1, x_2)$ represent the same preferences as $u(x_1, x_2)$? Justify your answer using the Marginal Rate of Substitution.
\end{enumerate}

\textbf{Solution:}

\textbf{Part 1: Calculate $MRS_u$}
\begin{enumerate}
    \item \textbf{Calculate Marginal Utilities for $u$:}
    \[ \text{MU}_1^u = \frac{\partial u}{\partial x_1} = \frac{1}{2}(x_1 x_2)^{-1/2} (x_2) = \frac{x_2}{2\sqrt{x_1 x_2}} \]
    \[ \text{MU}_2^u = \frac{\partial u}{\partial x_2} = \frac{1}{2}(x_1 x_2)^{-1/2} (x_1) = \frac{x_1}{2\sqrt{x_1 x_2}} \]
    \item \textbf{Calculate $MRS_u$:}
    \[ \text{MRS}_u = -\frac{\text{MU}_1^u}{\text{MU}_2^u} = -\frac{x_2 / (2\sqrt{x_1 x_2})}{x_1 / (2\sqrt{x_1 x_2})} = -\frac{x_2}{x_1} \]
\end{enumerate}

\textbf{Part 2: Determine Preference Equivalence with $v(x_1, x_2) = x_1^2 x_2^2$}
\begin{enumerate}
    \item \textbf{Calculate Marginal Utilities for $v$:}
    \[ \text{MU}_1^v = \frac{\partial v}{\partial x_1} = 2x_1 x_2^2 \]
    \[ \text{MU}_2^v = \frac{\partial v}{\partial x_2} = 2x_1^2 x_2 \]
    \item \textbf{Calculate $MRS_v$:}
    \[ \text{MRS}_v = -\frac{\text{MU}_1^v}{\text{MU}_2^v} = -\frac{2x_1 x_2^2}{2x_1^2 x_2} = -\frac{x_2}{x_1} \]
\end{enumerate}

\textbf{Conclusion:} Yes, $v(x_1, x_2)$ represents the same preferences as $u(x_1, x_2)$. Since $\text{MRS}_u = \text{MRS}_v = -x_2/x_1$, both utility functions define identical indifference curves (same slope everywhere) and thus represent the same underlying preference ordering.

\textit{(Self-Correction/Verification: $v(x_1, x_2) = x_1^2 x_2^2 = (x_1 x_2)^2$. Since $u(x_1, x_2) = \sqrt{x_1 x_2}$, we have $v = u^4$. Since $f(u)=u^4$ is a strictly monotonic transformation for $u \geq 0$, the preferences must be the same)}.

\hrulefill

\section*{2. Commodities and Utility (Continued)}

\subsection*{(c) Diminishing Marginal Rate of Substitution (MRS)}

The assumption that "averages are preferred to extremes" is formalized as \textbf{Convexity} of preferences. This characteristic is intrinsically linked to the behavior of the Marginal Rate of Substitution (MRS).

\textbf{Definition: Diminishing MRS and Convexity}

\begin{itemize}
    \item \textbf{Diminishing MRS:} As a consumer increases their consumption of good 1 (moving down and right along an indifference curve), the magnitude of the MRS decreases. This means the consumer is willing to give up less and less of good 2 to obtain an additional marginal unit of good 1.
    \item \textbf{Convexity Condition:} Preferences are strictly convex if and only if the MRS is strictly diminishing. Geometrically, this means indifference curves are bowed toward the origin.
    \begin{itemize}
        \item[$\circ$] \textbf{Note:} If the MRS is constant (linear ICs, e.g., Perfect Substitutes), preferences are convex, but not \textit{strictly} convex.
    \end{itemize}
\end{itemize}

\textbf{Mathematical Test for Diminishing MRS (Convexity)}

To verify that the MRS is diminishing, we examine how the MRS changes as $x_1$ increases along an indifference curve.

\begin{enumerate}
    \item \textbf{Recall MRS formula:} $\text{MRS} = \frac{dx_2}{dx_1} = -\frac{\text{MU}_1}{\text{MU}_2}$.
    \item \textbf{Test Condition (LaTeX):} Since the MRS is a negative number for goods, diminishing $|\text{MRS}|$ means the MRS value itself is becoming \textit{less negative} (i.e., increasing). $$\text{Diminishing MRS} \iff \frac{d|\text{MRS}|}{dx_1} < 0 \iff \frac{d(\text{MRS})}{dx_1} > 0$$ This is also equivalent to checking the sign of the second derivative of the indifference curve: $$\text{Convexity} \iff \frac{d^2x_2}{dx_1^2} > 0$$
\end{enumerate}

$\star$ \textbf{Exam Focus: Shortcuts \& Mistakes}
\begin{itemize}
    \item \textbf{Shortcut: Check Monotonicity First:} If a problem states preferences are monotonic, you know the MRS is negative. You only need to check the sign of the derivative of the MRS (or the second derivative of $x_2$) to determine convexity.
    \item \textbf{Pitfall: Concave Preferences:} If the second derivative $\frac{d^2x_2}{dx_1^2}$ is negative, the preferences are concave. Concave preferences imply that the consumer prefers extremes over mixtures, and will usually lead to corner solutions in optimization problems.
\end{itemize}

\textbf{Special Cases of Preferences and MRS Behavior}

The shape of the indifference curve directly determines the behavior of the MRS, providing a quick check for convexity classification.

\begin{center}
\begin{tabular}{lllll}
\toprule
\textbf{Preference Type} & \textbf{Indifference Curve Shape} & \textbf{MRS Behavior} & \textbf{Convexity} & \textbf{Source} \\
\midrule
\textbf{Perfect Substitutes} & Straight lines (constant slope) & Constant ($| \text{MRS}| = a/b$) & Convex & \\
\textbf{Perfect Complements} & L-shaped & Either 0 or $\infty$ (undefined at kink) & Convex & \\
\textbf{Cobb-Douglas} & Hyperbolic, bowed inward & Diminishing & Strictly Convex & \\
\textbf{Quasilinear} & Parallel shifts (vertical/horizontal) & Depends on one good only & Diminishing (usually) & \\
\bottomrule
\end{tabular}
\end{center}

\hrule

\textbf{Step-by-Step Worked Example: Cobb-Douglas Preferences}

Verify that the utility function $u(x_1, x_2) = x_1 x_2$ exhibits diminishing MRS (i.e., is convex).

\textbf{Step 1: Calculate Marginal Utilities (MU)} $$ MU_1 = \frac{\partial u}{\partial x_1} = x_2 $$ $$ MU_2 = \frac{\partial u}{\partial x_2} = x_1 $$

\textbf{Step 2: Calculate the Marginal Rate of Substitution (MRS)} $$ \text{MRS} = -\frac{MU_1}{MU_2} = -\frac{x_2}{x_1} $$

\textbf{Step 3: Solve for $x_2$ along the indifference curve ($u(x_1, x_2) = k$)} $$ x_1 x_2 = k \implies x_2 = \frac{k}{x_1} $$

\textbf{Step 4: Substitute $x_2$ into the MRS formula (if needed), and check the second derivative $\frac{d^2x_2}{dx_1^2}$} Alternatively, since we know $x_2$ decreases as $x_1$ increases (to keep $k$ constant) and $|\text{MRS}| = x_2/x_1$: As $x_1$ increases:

\begin{enumerate}
    \item $x_1$ increases (denominator grows).
    \item $x_2$ decreases (numerator shrinks).
    \item Therefore, the magnitude $|\text{MRS}|$ shrinks, meaning the MRS is diminishing.
\end{enumerate}

Formally, using $\frac{d^2x_2}{dx_1^2}$ (which must be $> 0$ for convexity): $$ \frac{dx_2}{dx_1} = \text{MRS} = -k x_1^{-2} $$ $$ \frac{d^2x_2}{dx_1^2} = \frac{d}{dx_1}(-k x_1^{-2}) = (-k)(-2) x_1^{-3} = \frac{2k}{x_1^3} $$ Since $k$ (utility level) and $x_1$ (consumption) are positive, $\frac{2k}{x_1^3} > 0$.

\textbf{Conclusion:} The second derivative is positive, confirming the MRS is diminishing, and the preferences are strictly convex.

\hrule

\textbf{Exam-Style Problem 6: Classifying Preferences by MRS Behavior}

\textbf{Problem (Based on PS1 Q5 \& PE Q6):}

The utility function for a consumer is given by $u(x_1, x_2) = 4x_1^2 + x_2^2$.

\begin{enumerate}
    \item Calculate the Marginal Rate of Substitution (MRS) in terms of $x_1$ and $x_2$.
    \item Determine if the preferences are convex or concave by checking the behavior of the MRS.
\end{enumerate}

\textbf{Solution:}

\textbf{Part 1: Calculate MRS}
\begin{enumerate}
    \item \textbf{Calculate Marginal Utilities (MU):} $$ MU_1 = \frac{\partial u}{\partial x_1} = 8x_1 $$ $$ MU_2 = \frac{\partial u}{\partial x_2} = 2x_2 $$
    \item \textbf{Calculate MRS:} $$ \text{MRS} = -\frac{MU_1}{MU_2} = -\frac{8x_1}{2x_2} = -\frac{4x_1}{x_2} $$
\end{enumerate}

\textbf{Part 2: Determine Convexity/Concavity}

\textbf{Strategy:} Determine if the magnitude $|\text{MRS}| = 4x_1/x_2$ increases or decreases as $x_1$ increases along an indifference curve ($u(x_1, x_2)=k$).

\begin{enumerate}
    \item \textbf{Solve for $x_2$ along $u(x_1, x_2) = k$:} $$ 4x_1^2 + x_2^2 = k \implies x_2 = \sqrt{k - 4x_1^2} $$
    \item \textbf{Analyze the relationship between $x_1$ and $x_2$:} As $x_1$ increases, the term $4x_1^2$ increases, meaning $x_2$ must \textbf{decrease} to keep $k$ constant.
    \item \textbf{Analyze the magnitude of the MRS:} $$ |\text{MRS}| = 4 \frac{x_1}{x_2} $$ Since $x_1$ is increasing and $x_2$ is decreasing, the ratio $x_1/x_2$ must \textbf{increase}.
    \item \textbf{Conclusion:} The magnitude $|\text{MRS}|$ is increasing as $x_1$ increases. This means the MRS is \textbf{increasing} (not diminishing), indicating that the preferences are \textbf{concave}. The consumer prefers extremes (corner solutions) over mixtures.
\end{enumerate}

\hrule

\section*{Budget Constraint}

\subsection*{Definition}

The \textbf{budget constraint} defines the limits on consumption choices imposed by prices and income. A consumer aims to choose the best bundle of goods they can afford.

\textbf{Clear Definitions of Key Concepts}

\begin{itemize}
    \item \textbf{Consumption Bundle:} A specific combination of goods, represented by a vector $x = (x_1, x_2)$ (in the two-good case).
    \item \textbf{Price ($p_i$):} The price per unit of good $i$. Prices $p_1$ and $p_2$ are assumed fixed for the consumer.
    \item \textbf{Income ($m$):} The total amount of money the consumer has to spend.
    \item \textbf{Budget Set:} The set of all consumption bundles $(x_1, x_2)$ that are affordable given prices and income.
    \item \textbf{Budget Line:} The set of consumption bundles for which the consumer spends \textit{exactly} their entire income.
\end{itemize}

\textbf{Relevant Formulas and Derivations (LaTeX)}

\begin{enumerate}
    \item \textbf{Budget Constraint (Inequality):} $$\text{Expenditure} \leq \text{Income} \implies p_1 x_1 + p_2 x_2 \leq m$$.
    \item \textbf{Budget Line (Equality):} $$p_1 x_1 + p_2 x_2 = m$$.
    \item \textbf{Budget Line in Slope-Intercept Form:} The budget line rearranged to express consumption of good 2 as a function of good 1: $$x_2 = \frac{m}{p_2} - \frac{p_1}{p_2} x_1$$.
    \item \textbf{Properties of the Budget Line:}
    \begin{itemize}
        \item \textbf{Slope:} The rate at which good 2 must be given up to purchase one more unit of good 1. $$\text{Slope} = -\frac{p_1}{p_2}$$.
        \item \textbf{Vertical Intercept:} Maximum amount of good 2 that can be purchased ($\text{set } x_1 = 0$). $$\text{Vertical Intercept} = \frac{m}{p_2}$$.
        \item \textbf{Horizontal Intercept:} Maximum amount of good 1 that can be purchased ($\text{set } x_2 = 0$). $$\text{Horizontal Intercept} = \frac{m}{p_1}$$.
    \end{itemize}
\end{enumerate}

\textbf{$\star$ Exam Focus: Shortcuts \& Pitfalls}

\begin{itemize}
    \item \textbf{Shortcut: Numeraire (Scaling):} Multiplying all prices ($p_1, p_2$) and income ($m$) by any positive constant $t$ yields the same budget set and budget line. Thus, only relative prices ($\frac{p_1}{p_2}$) matter for the slope, and real income ($\frac{m}{p_i}$) matters for the intercepts.
    \item \textbf{Interpretation Pitfall (Slope):} The magnitude of the slope, $|-p_1/p_2| = p_1/p_2$, represents the \textbf{opportunity cost} of good 1 in terms of good 2 (the amount of good 2 the market requires the consumer to give up for one more unit of good 1).
    \item \textbf{Conceptual Check:} A perfectly balanced inflation where $p_1, p_2,$ and $m$ all rise by the same proportion does \textbf{not} change the budget set, and thus does not change the optimal choice.
\end{itemize}

\textbf{Step-by-Step Worked Example: Calculating Income and Price Ratios}

A consumer observes that the bundles $A=(10, 5)$ and $B=(6, 15)$ lie exactly on their budget line. Determine the relative price ratio ($p_1/p_2$) and the total income ($m$) expressed in terms of $p_2$.

\textbf{Step 1: Calculate the slope of the budget line.} The slope is $\Delta x_2 / \Delta x_1$ between any two points on the line. $$\text{Slope} = \frac{15 - 5}{6 - 10} = \frac{10}{-4} = -2.5$$

\textbf{Step 2: Use the slope to find the relative price ratio ($p_1/p_2$).} We know that the slope equals $-p_1/p_2$: $$-\frac{p_1}{p_2} = -2.5 \implies \frac{p_1}{p_2} = 2.5$$ (Interpretation: Good 1 is 2.5 times more expensive than Good 2).

\textbf{Step 3: Express $p_1$ in terms of $p_2$.} $$p_1 = 2.5 p_2$$

\textbf{Step 4: Use the budget line equation and a chosen bundle (A) to solve for $m$ in terms of $p_2$.} Using $p_1x_1 + p_2x_2 = m$ and Bundle $A=(10, 5)$: $$m = p_1(10) + p_2(5)$$ Substitute $p_1 = 2.5 p_2$: $$m = (2.5 p_2)(10) + p_2(5)$$ $$m = 25 p_2 + 5 p_2 = 30 p_2$$

\textbf{Conclusion:} The price ratio $p_1/p_2$ is $2.5$, and the income $m$ is equivalent to $30 p_2$. If $p_2=1$, $p_1=2.5$ and $m=30$. (Note: If the consumer spent all income on Good 2, they could buy $m/p_2 = 30$ units.)

\textbf{Exam-Style Problem 7: Budget Line under Taxes and Subsidies}

\textbf{Problem (Adapted from Varian Review Q2.6):}

Suppose a budget equation is given by $p_1x_1 + p_2x_2 = m$. The government decides to impose a lump-sum tax of $u$, a quantity tax on good 1 of $t$, and a quantity subsidy on good 2 of $s$.

What is the formula for the new budget line?

\textbf{Solution:}

\begin{enumerate}
    \item \textbf{Identify the effective price of Good 1:} A quantity tax $t$ raises the price paid by the consumer. The new effective price is $(p_1 + t)$.
    \item \textbf{Identify the effective price of Good 2:} A quantity subsidy $s$ lowers the price paid by the consumer. The new effective price is $(p_2 - s)$.
    \item \textbf{Identify the effective income:} A lump-sum tax $u$ reduces the consumer's total purchasing power (income). The new effective income is $(m - u)$.
    \item \textbf{Construct the new budget line formula:} Replace the original prices and income with the effective values [76, Answer A12].
\end{enumerate}

\textbf{Formula (LaTeX):} $$(p_1 + t)x_1 + (p_2 - s)x_2 = m - u$$.

\textit{This represents a budget line that has rotated (due to changes in $p_1$ and $p_2$) and shifted inward (due to the lump-sum tax $u$).}

\hrulefill

\section*{3. Budget Constraint (Continued)}
\subsection*{(b) Shifts and Rotations}

The position and orientation of the budget line are determined entirely by the set of prices ($p_1, p_2$) and income ($m$). Changes in these parameters cause predictable shifts or rotations of the boundary of the budget set.

\subsubsection*{Mechanisms of Change}

The consumer's budget line can be expressed in the slope-intercept form: \[ x_2 = \frac{m}{p_2} - \frac{p_1}{p_2} x_1 \]

\centering
\begin{tabular}{>{\raggedright\arraybackslash}p{0.18\linewidth} >{\raggedright\arraybackslash}p{0.28\linewidth} >{\raggedright\arraybackslash}p{0.24\linewidth} >{\raggedright\arraybackslash}p{0.20\linewidth} p{0cm}}
\toprule
\textbf{Parameter Change}& \textbf{Effect on Intercepts}& \textbf{Effect on Slope}& \textbf{Resulting Geometric Change}& \textbf{Source} \\
\midrule
\textbf{Income ($\uparrow m$)}&  Both intercepts ($\frac{m}{p_1}, \frac{m}{p_2}$) increase.& Unchanged ($-\frac{p_1}{p_2}$)& Parallel shift outward&   \\ \addlinespace
\textbf{Price 1 ($\uparrow p_1$)}&  Vertical intercept ($\frac{m}{p_2}$) unchanged; Horizontal intercept ($\frac{m}{p_1}$) shrinks (moves inward).& Becomes steeper (more negative, $| -\frac{p_1}{p_2}| $ increases). & & \\ \addlinespace
\textbf{Price 2 ($\downarrow p_2$)}&  Horizontal intercept ($\frac{m}{p_1}$) unchanged; Vertical intercept ($\frac{m}{p_2}$) expands (moves outward).& Becomes flatter (less negative, $| -\frac{p_1}{p_2}| $ decreases). & & \\
\bottomrule
\end{tabular}

\subsubsection*{$\star$ Exam Focus: Shortcuts \& Pitfalls}
\begin{itemize}
    \item \textbf{Shortcut: The Numeraire Principle:} Multiplying all prices ($p_1, p_2$) and income ($m$) by the same positive number ($t > 0$) leaves the budget constraint unchanged. This is because the price ratios and the real income (purchasing power) remain the same.
    \begin{itemize}
        \item[$\circ$] \textit{Example:} $p_1 x_1 + p_2 x_2 = m$ is the same budget line as $tp_1 x_1 + tp_2 x_2 = tm$.
    \end{itemize}
    \item \textbf{Pitfall: Distinguishing Shifts from Rotations:} A change in income ($m$) changes purchasing power uniformly and causes a parallel shift. A change in a single price ($p_i$) changes both purchasing power (real income) and relative prices (slope) and causes a rotation (pivot).
\end{itemize}

\subsubsection*{Step-by-Step Worked Example: Combined Price and Income Changes}

A consumer starts with $p_1=5$, $p_2=5$, and $m=100$. The price of good 1 decreases to $p_1'=4$, and income simultaneously increases to $m'=120$. Analyze the change in the budget line.

\textbf{Step 1: Determine Initial Budget Line Parameters} \[ \text{Initial Slope} = -\frac{p_1}{p_2} = -\frac{5}{5} = -1 \] \[ \text{Initial Horizontal Intercept} = \frac{m}{p_1} = \frac{100}{5} = 20 \] \[ \text{Initial Vertical Intercept} = \frac{m}{p_2} = \frac{100}{5} = 20 \]

\textbf{Step 2: Determine Final Budget Line Parameters} \[ \text{Final Slope} = -\frac{p_1'}{p_2'} = -\frac{4}{5} = -0.8 \] \[ \text{Final Horizontal Intercept} = \frac{m'}{p_1'} = \frac{120}{4} = 30 \] \[ \text{Final Vertical Intercept} = \frac{m'}{p_2'} = \frac{120}{5} = 24 \]

\textbf{Step 3: Analyze Geometric Change}
\begin{itemize}
    \item \textbf{Slope:} The slope changes from $-1$ to $-0.8$. The budget line becomes \textbf{flatter} (the opportunity cost of $x_1$ decreased).
    \item \textbf{Intercepts:} Both intercepts increased (H-intercept $20 \to 30$; V-intercept $20 \to 24$).
    \item \textbf{Conclusion:} The budget line shifts \textbf{outward} (increased budget set) and \textbf{rotates} (becomes flatter).
\end{itemize}
\noindent\rule{\linewidth}{0.4pt}

\subsubsection*{Exam-Style Problem 8: Analyzing Relative Price Changes}

\textbf{Problem (Based on Varian Q 2.3):}
\begin{enumerate}
    \item If the price of good 1 doubles ($2p_1$) and the price of good 2 triples ($3p_2$), does the new budget line become flatter or steeper compared to the original line?
    \item A consumer faces $p_1=10$ and $p_2=5$. If both prices and income ($m$) double, what happens to the consumer's budget set? What is the new opportunity cost of good 1?
\end{enumerate}

\textbf{Solution:}

\textbf{Part 1: Relative Steepness (Slope Comparison)}
\begin{enumerate}
    \item \textbf{Original Slope:} $\text{Slope}_{\text{Orig}} = -\frac{p_1}{p_2}$
    \item \textbf{New Slope:} $\text{Slope}_{\text{New}} = -\frac{2p_1}{3p_2} = -\frac{2}{3} \cdot \frac{p_1}{p_2}$
    \item \textbf{Comparison:} Since $\frac{2}{3} < 1$, the magnitude of the new slope is less than the magnitude of the original slope.
    \item \textbf{Conclusion:} The new budget line is \textbf{flatter}. (Since $p_2$ increased proportionally more than $p_1$, good 1 became relatively cheaper).
\end{enumerate}

\textbf{Part 2: Proportional Changes (Numeraire)}
\begin{enumerate}
    \item \textbf{Original Budget Line:} $10x_1 + 5x_2 = m$
    \item \textbf{New Parameters (Doubled):} $p_1'=20$, $p_2'=10$, $m'=2m$.
    \item \textbf{New Budget Line:} $20x_1 + 10x_2 = 2m$.
    \item \textbf{Analysis:} If we divide the new budget line by 2 (the scale factor), we get $10x_1 + 5x_2 = m$. This is the exact same budget line as the original.
    \item \textbf{Conclusion on Budget Set:} The budget set \textbf{remains unchanged}.
    \item \textbf{New Opportunity Cost:} The opportunity cost of good 1 is $|-\frac{p_1'}{p_2'}| = \frac{20}{10} = 2$.
    \begin{itemize}
        \item[$\circ$] Since the relative price ratio $p_1/p_2$ is unchanged, the opportunity cost remains 2.
    \end{itemize}
\end{enumerate}
\noindent\rule{\linewidth}{0.4pt}

\section{Consumer Choice}
\subsection{Interior Optimum}

The core of consumer theory is solving the \textbf{Consumer's Problem}: maximizing utility subject to the budget constraint.

\[\max_{x_1, x_2} u(x_1, x_2) \quad \text{subject to} \quad p_1 x_1 + p_2 x_2 \leq m\]

An \textbf{Interior Solution} is achieved when the optimal bundle, $(x_1^*, x_2^*)$, involves strictly positive consumption of both goods ($x_1^* > 0$ and $x_2^* > 0$). For preferences that are monotonic and smooth, an interior optimum satisfies the \textbf{Tangency Condition}.

\noindent Relevant Formulas and Conditions (LaTeX)
\begin{enumerate}
    \item \textbf{Tangency Condition (MRS equals Price Ratio):} The rate at which the consumer is willing to trade the goods (MRS) must equal the rate at which the market allows trade (price ratio).
    \[\text{MRS} = -\frac{p_1}{p_2} \iff \frac{\text{MU}_1}{\text{MU}_2} = \frac{p_1}{p_2}\]
    where $\text{MU}_i = \partial u / \partial x_i$ is the Marginal Utility of good $i$.

    \item \textbf{Budget Constraint (Binding):} Due to the Monotonicity assumption (more is better), the consumer must spend all available income at the optimum.
    \[p_1 x_1 + p_2 x_2 = m\]
\end{enumerate}

\noindent Step-by-Step Derivation: Why $\text{MRS} = -p_1/p_2$

The equality condition ensures that there is no scope for profitable deviation by trading goods on the market.
\begin{enumerate}
    \item \textbf{Calculus Approach (Substitution Method):} Since the optimum must lie on the budget line (due to monotonicity), we can substitute the budget line into the utility function to turn the constrained optimization problem into an unconstrained one.
    \[x_2 = \frac{m}{p_2} - \frac{p_1}{p_2} x_1\]
    \[\max_{x_1} u\left(x_1, \frac{m}{p_2} - \frac{p_1}{p_2} x_1\right)\]

    \item \textbf{First-Order Condition (FOC):} Differentiate the maximized utility function with respect to $x_1$ and set the result to zero. (Using the chain rule, where $\frac{dx_2}{dx_1} = -\frac{p_1}{p_2}$):
    \[\frac{\partial u}{\partial x_1} + \frac{\partial u}{\partial x_2} \cdot \left( -\frac{p_1}{p_2} \right) = 0\]

    \item \textbf{Rearrangement:} Rearranging this FOC yields the tangency condition:
    \[\frac{\partial u / \partial x_1}{\partial u / \partial x_2} = \frac{p_1}{p_2} \implies \frac{\text{MU}_1}{\text{MU}_2} = \frac{p_1}{p_2}\]
\end{enumerate}

$\star$ \textbf{Exam Focus: Shortcuts \& Mistakes}
\begin{itemize}
    \item \textbf{Necessity vs. Sufficiency:} The tangency condition ($\text{MRS} = -p_1/p_2$) is a \textbf{necessary condition} for an interior optimum. It is a \textbf{sufficient condition} \textit{only if} preferences are \textbf{convex}. For concave preferences, tangency usually indicates a local minimum, and the optimum will be a corner solution.

    \item \textbf{MRS Interpretation Pitfall:} If $|\text{MRS}|$ is the rate of exchange the consumer is \textit{willing} to accept, and $p_1/p_2$ is the rate the market \textit{offers}:
    \begin{itemize}
        \item If $|\text{MRS}| > p_1/p_2$, the consumer is willing to give up more $x_2$ for $x_1$ than the market requires; consume more $x_1$ (move along the budget line).
        \item If $|\text{MRS}| < p_1/p_2$, the consumer should consume less $x_1$ and more $x_2$.
    \end{itemize}
\end{itemize}

\vspace{0.5em}
\hrulefill
\vspace{0.5em}

\noindent\textbf{Step-by-Step Worked Example: Cobb-Douglas Optimization}

Find the optimal bundle for the utility function $u(x_1, x_2) = x_1^2 x_2$, given prices $p_1=4$, $p_2=1$, and income $m=180$. (We assume these preferences are convex, which Cobb-Douglas preferences generally are, guaranteeing the tangency yields the solution.)

\textbf{Step 1: Calculate Marginal Utilities and MRS}
\[ MU_1 = \frac{\partial u}{\partial x_1} = 2x_1 x_2 \]
\[ MU_2 = \frac{\partial u}{\partial x_2} = x_1^2 \]
\[ \text{MRS} = -\frac{MU_1}{MU_2} = -\frac{2x_1 x_2}{x_1^2} = -\frac{2x_2}{x_1} \]

\textbf{Step 2: Apply the Tangency Condition}
Set $|\text{MRS}|$ equal to the price ratio $p_1/p_2 = 4/1 = 4$:
\[ \frac{2x_2}{x_1} = 4 \implies 2x_2 = 4x_1 \implies x_2 = 2x_1 \]
\textit{(This is the optimization path/relationship between the two goods).}

\textbf{Step 3: Apply the Budget Constraint}
Substitute the relationship from Step 2 into the budget equation $p_1 x_1 + p_2 x_2 = m$:
\[ 4x_1 + 1(2x_1) = 180 \]
\[ 6x_1 = 180 \]
\[ x_1^* = 30 \]

\textbf{Step 4: Find $x_2^*$}
Use the relationship from Step 2:
\[ x_2^* = 2x_1^* = 2(30) = 60 \]

\textbf{Optimal Bundle:} $(x_1^*, x_2^*) = (30, 60)$.

\vspace{0.5em}
\hrulefill
\vspace{0.5em}

\noindent\textbf{Exam-Style Problem 9: Cobb-Douglas Utility (Logarithmic Form)}

Find the optimal consumption bundle for a consumer with utility function $u(x_1, x_2) = \ln x_1 + \ln x_2$, prices $p_1=10$ and $p_2=5$, and income $m=150$.

\textbf{Solution Strategy:} Since $u(x_1, x_2) = \ln x_1 + \ln x_2$ is a monotonic transformation of $v(x_1, x_2) = x_1 x_2$ (Cobb-Douglas), the preferences are strictly convex, guaranteeing an interior solution using the tangency condition.

\textbf{Step 1: Calculate Marginal Utilities and MRS}
\[ MU_1 = \frac{1}{x_1}, \quad MU_2 = \frac{1}{x_2} \]
\[ \text{MRS} = -\frac{MU_1}{MU_2} = -\frac{1/x_1}{1/x_2} = -\frac{x_2}{x_1} \]

\textbf{Step 2: Apply the Tangency Condition}
Set $|\text{MRS}|$ equal to the price ratio $p_1/p_2 = 10/5 = 2$:
\[ \frac{x_2}{x_1} = 2 \implies x_2 = 2x_1 \]

\textbf{Step 3: Apply the Budget Constraint}
Substitute $x_2 = 2x_1$ into the budget equation $10x_1 + 5x_2 = 150$:
\[ 10x_1 + 5(2x_1) = 150 \]
\[ 10x_1 + 10x_1 = 150 \]
\[ 20x_1 = 150 \implies x_1^* = 7.5 \]

\textbf{Step 4: Find $x_2^*$}
\[ x_2^* = 2x_1^* = 2(7.5) = 15 \]

\textbf{Optimal Bundle:} $(x_1^*, x_2^*) = (7.5, 15)$.

\hrulefill

\section{Consumer Choice (Continued)}

\subsection{Corner Solutions}

A \textbf{Corner Solution} (or \textbf{Boundary Optimum}) occurs when the optimal consumption bundle $(x_1^*, x_2^*)$ involves consuming none of a specific good. This is defined by $x_i^* = 0$ for at least one good $i$.

\subsubsection*{When Corner Solutions Occur}

\begin{enumerate}
    \item \textbf{Non-Convex Preferences:} If preferences are strictly concave (bowed away from the origin), the consumer always prefers extreme bundles over mixtures, leading invariably to a corner solution.
    \item \textbf{Perfect Substitutes:} Since ICs are linear, the optimal choice is a corner solution unless the MRS exactly equals the price ratio.
    \item \textbf{Interior Fails (Quasilinear/Extreme Prices):} Even with strictly convex preferences, if the tangency condition yields a result where $x_i < 0$, that bundle is infeasible, necessitating a corner solution.
\end{enumerate}

\subsubsection*{Condition for Corner Optimum (Convex Preferences)}

If the optimal choice is a corner solution for good 1 ($x_1^*=0$), the utility maximization criterion requires that at this point, the consumer's willingness to trade ($|\text{MRS}|$) is less than or equal to the market trade rate ($\frac{p_1}{p_2}$).
\[ \text{At } x_1^* =0, x_2^*=\frac{m}{p_2}: \quad |\text{MRS}| \leq \frac{p_1}{p_2} \]
If $|\text{MRS}| > \frac{p_1}{p_2}$ at $x_1^*=0$, the consumer would gain utility by moving away from the corner to consume some $x_1$, meaning the corner solution cannot be optimal.

\subsubsection*{Step-by-Step Problem Solving Strategy for Corner Solutions}

If the initial MRS condition (Step 1 below) yields an infeasible negative consumption value, or if preferences are non-Cobb-Douglas/non-Perfect Complements, use this explicit comparison:

\begin{enumerate}
    \item \textbf{Attempt Interior Solution:} Set $|\text{MRS}| = \frac{p_1}{p_2}$. Solve for $x_1$ and $x_2$.
    \item \textbf{Check Feasibility:} If the resulting $(x_1, x_2)$ requires $x_i < 0$, reject the interior solution.
    \item \textbf{Identify Corner Candidates:} The two candidates for the optimum are:
    \begin{itemize}
        \item Corner 1: $(x_1^C, x_2^C) = (\frac{m}{p_1}, 0)$
        \item Corner 2: $(x_1^D, x_2^D) = (0, \frac{m}{p_2})$
    \end{itemize}
    \item \textbf{Compare Utility:} Choose the bundle that yields the maximum utility: \[\text{Optimal Bundle } x^* \in \arg\max \left\{ u\left(\frac{m}{p_1}, 0\right), u\left(0, \frac{m}{p_2}\right) \right\}\]
\end{enumerate}

\noindent\rule{\linewidth}{0.4pt}

\subsubsection*{$\star$ Exam Focus: Shortcuts \& Mistakes}

\begin{itemize}
    \item \textbf{Mistake: Stopping at Negative:} If the tangency condition yields $x_1=-3$, DO NOT panic and DO NOT stop. This is merely confirmation that the optimal choice is a corner solution, usually $x_1^*=0$.
    \item \textbf{Shortcut for Perfect Substitutes ($u=ax_1 + bx_2$):} Compare the marginal utility per dollar spent immediately, without resorting to the general utility comparison method:
    \begin{itemize}
        \item If $\frac{a}{p_1} > \frac{b}{p_2}$, buy only $x_1$: $x^* = (\frac{m}{p_1}, 0)$.
        \item If $\frac{a}{p_1} < \frac{b}{p_2}$, buy only $x_2$: $x^* = (0, \frac{m}{p_2})$.
        \item If $\frac{a}{p_1} = \frac{b}{p_2}$, any bundle on the budget line is optimal.
    \end{itemize}
\end{itemize}

\noindent\rule{\linewidth}{0.4pt}

\subsubsection*{Step-by-Step Worked Example: Perfect Substitutes}

Find the optimal consumption bundle for $u(x_1, x_2) = x_1 + 3x_2$ given prices $p_1=2$, $p_2=5$, and income $m=30$.

\textbf{Step 1: Identify Preference Type and MRS} This is a Perfect Substitutes utility function (linear ICs). $MU_1 = 1$, $MU_2 = 3$. $|\text{MRS}| = \frac{MU_1}{MU_2} = \frac{1}{3}$. The market price ratio is $\frac{p_1}{p_2} = \frac{2}{5} = 0.4$.

\textbf{Step 2: Compare Willingness to Trade to Market Rate} The consumer is willing to trade $\frac{1}{3} \approx 0.33$ units of $x_2$ for one unit of $x_1$. The market requires the consumer to trade $\frac{2}{5} = 0.4$ units of $x_2$ for one unit of $x_1$. Since $|\text{MRS}| < \frac{p_1}{p_2}$, the indifference curve is \textit{flatter} than the budget line. The consumer values $x_1$ less than the market demands, so they should specialize in the relatively preferred good, $x_2$.

\textit{Shortcut Check: Compare utility per dollar spent:}
\[ \frac{MU_1}{p_1} = \frac{1}{2} = 0.5 \quad \text{vs.} \quad \frac{MU_2}{p_2} = \frac{3}{5} = 0.6 \]
Since $0.6 > 0.5$, the marginal utility received per dollar spent is higher for good 2. Spend all income on good 2.

\textbf{Step 3: Determine Optimal Corner Solution} The consumer buys $x_1^*=0$.
\[ p_1 x_1^* + p_2 x_2^* = m \implies 2(0) + 5x_2^* = 30 \]
\[ x_2^* = 6 \]
\textbf{Optimal Bundle:} $(x_1^*, x_2^*) = (0, 6)$.

\noindent\rule{\linewidth}{0.4pt}

\subsubsection*{Exam-Style Problem 10: Concave Preferences}

\textbf{Problem (Corner Solution Necessary):}

The consumer has the utility function $u(x_1, x_2) = 2x_1^2 + x_2^2$. This preference is known to be strictly concave. Find the optimal consumption bundle given prices $p_1=10$, $p_2=5$, and income $m=50$.

\textbf{Solution:}

\textbf{Strategy:} Since preferences are strictly concave, the optimal solution must be a corner solution, as the consumer prefers extremes over mixtures. We compare the utility yielded by the two possible corner bundles.

\textbf{Step 1: Define Corner Candidates}
\begin{enumerate}
    \item \textbf{Corner 1 (Only Good 1):} Spend all $m$ on $x_1$.
    \[ x_1^C = \frac{m}{p_1} = \frac{50}{10} = 5, \quad x_2^C = 0 \]
    \[ \text{Utility } u(5, 0) = 2(5^2) + 0^2 = 50 \]
    \item \textbf{Corner 2 (Only Good 2):} Spend all $m$ on $x_2$.
    \[ x_1^D = 0, \quad x_2^D = \frac{m}{p_2} = \frac{50}{5} = 10 \]
    \[ \text{Utility } u(0, 10) = 2(0^2) + 10^2 = 100 \]
\end{enumerate}

\textbf{Step 2: Compare Utilities} Since $u(0, 10) = 100$ is greater than $u(5, 0) = 50$, the consumer maximizes utility by choosing Corner 2.

\textbf{Optimal Bundle:} $(x_1^*, x_2^*) = (0, 10)$.

\textit{(Self-Check: If we attempted the tangency condition: $|\text{MRS}| = \frac{4x_1}{2x_2} = \frac{2x_1}{x_2}$. Set $\frac{2x_1}{x_2} = \frac{p_1}{p_2} = \frac{10}{5} = 2 \implies x_1 = x_2$. Budget: $10x_1 + 5x_1 = 50 \implies 15x_1 = 50 \implies x_1 \approx 3.33$. Utility is $2(3.33)^2 + (3.33)^2 \approx 33.33$. Since $100 > 33.33$, the interior tangency is a local minimum, confirming the corner solution is superior.)}

\noindent\rule{\linewidth}{0.4pt}

\section{Consumer Choice (Continued)}
\subsection{Special Preferences}

While the Tangency Condition ($\text{MRS} = -p_1/p_2$) is the central solution method for interior optima with strictly convex preferences (like Cobb-Douglas), specific preference types require alternative optimization strategies.

\noindent\textbf{I. Perfect Substitutes}
\begin{itemize}
    \item \textbf{Definition:} The consumer is willing to substitute goods at a constant rate, $a:b$. Indifference curves are straight lines with a constant slope of $-\frac{a}{b}$.
    \item \textbf{Utility Function (LaTeX):} \[u(x_1, x_2) = a x_1 + b x_2\]
    \item \textbf{MRS:} The marginal rate of substitution is constant: \[\text{MRS} = -\frac{a}{b}\]
\end{itemize}

\noindent\textbf{Optimization Strategy: Corner Solutions}

Since the ICs are straight lines, if the slope of the IC ($\text{MRS}$) differs from the slope of the budget line ($-p_1/p_2$), the optimum will always be a \textbf{corner solution} (unless the slopes are identical).

\begin{enumerate}
    \item \textbf{Compare Utility per Dollar Spent:} Check the marginal utility per dollar for each good.
    \begin{itemize}
        \item \textbf{Formula (LaTeX):} Compare $\frac{MU_1}{p_1}$ vs. $\frac{MU_2}{p_2}$, where $MU_1=a$ and $MU_2=b$.
        \item If $\frac{a}{p_1} > \frac{b}{p_2}$: Spend all income on $x_1$. Optimal bundle is $\left(\frac{m}{p_1}, 0\right)$.
        \item If $\frac{a}{p_1} < \frac{b}{p_2}$: Spend all income on $x_2$. Optimal bundle is $\left(0, \frac{m}{p_2}\right)$.
        \item If $\frac{a}{p_1} = \frac{b}{p_2}$: Any bundle on the budget line is optimal.
    \end{itemize}
\end{enumerate}

\medskip\noindent\rule{\linewidth}{0.4pt}\medskip

\noindent\textbf{II. Perfect Complements}
\begin{itemize}
    \item \textbf{Definition:} Goods that are always consumed together in fixed proportions, $b$ units of $x_1$ for $a$ units of $x_2$. Indifference curves are L-shaped.
    \item \textbf{Utility Function (LaTeX):} \[u(x_1, x_2) = \min\{a x_1, b x_2\}\]
\end{itemize}

\noindent\textbf{Optimization Strategy: Fixed Proportion}

The optimal bundle must occur at the vertex (kink) of the L-shaped indifference curve, where the consumer avoids wasting either good.

\begin{enumerate}
    \item \textbf{Fixed Proportion Condition:} Set the arguments inside the minimum function equal: \[a x_1 = b x_2 \implies x_2 = \frac{a}{b} x_1\]
    \item \textbf{Budget Constraint:} Substitute this proportion into the budget equation to solve for $x_1^*$ and $x_2^*$: \[p_1 x_1 + p_2 \left(\frac{a}{b} x_1\right) = m\] \[x_1^* = \frac{m}{p_1 + p_2(a/b)}\]
\end{enumerate}

\medskip\noindent\rule{\linewidth}{0.4pt}\medskip

\subsubsection*{$\star$ Exam Focus: Shortcuts \& Mistakes}
\begin{itemize}
    \item \textbf{Perfect Substitutes Shortcut:} When testing equivalence of utility functions for perfect substitutes, check if the ratio of coefficients ($a/b$) is the same. The MRS determines preference, not the individual magnitudes of $a$ or $b$.
    \item \textbf{Perfect Complements Mistake:} Do not attempt to use the MRS for optimization! The MRS is undefined or discontinuous at the optimal kink point, meaning the tangency condition cannot be used.
\end{itemize}

\medskip\noindent\rule{\linewidth}{0.4pt}\medskip

\subsubsection*{Step-by-Step Worked Example: Perfect Substitutes}

Find the optimal consumption bundle for $u(x_1, x_2) = 2x_1 + 3x_2$ given prices $p_1=10$, $p_2=5$, and income $m=50$.

\textbf{Step 1: Calculate Marginal Utility per Dollar for each good}
\[ \frac{MU_1}{p_1} = \frac{2}{10} = 0.2 \]
\[ \frac{MU_2}{p_2} = \frac{3}{5} = 0.6 \]

\textbf{Step 2: Compare and Determine Specialization}
Since $\frac{MU_2}{p_2} (0.6)$ is strictly greater than $\frac{MU_1}{p_1} (0.2)$, the consumer maximizes utility by specializing entirely in good 2.
\[ x_1^* = 0 \]

\textbf{Step 3: Calculate $x_2^*$ using the Budget Constraint}
\[ p_1 x_1^* + p_2 x_2^* = m \]
\[ 10(0) + 5x_2^* = 50 \]
\[ x_2^* = 10 \]

\textbf{Optimal Bundle:} $(x_1^*, x_2^*) = (0, 10)$.

\medskip\noindent\rule{\linewidth}{0.4pt}\medskip

\subsubsection*{Step-by-Step Worked Example: Perfect Complements}

Find the optimal consumption bundle for $u(x_1, x_2) = \min\{4x_1, x_2\}$ given prices $p_1=3$, $p_2=2$, and income $m=44$.

\textbf{Step 1: Set the fixed consumption ratio}
The consumer consumes $x_1$ and $x_2$ such that the inputs are in the proper proportion:
\[ 4x_1 = x_2 \]

\textbf{Step 2: Substitute into the Budget Constraint}
The budget constraint is $3x_1 + 2x_2 = 44$. Substitute $x_2 = 4x_1$:
\[ 3x_1 + 2(4x_1) = 44 \]
\[ 3x_1 + 8x_1 = 44 \]
\[ 11x_1 = 44 \]

\textbf{Step 3: Solve for $x_1^*$ and $x_2^*$}
\[ x_1^* = 4 \]
Using the ratio:
\[ x_2^* = 4(x_1^*) = 4(4) = 16 \]

\textbf{Optimal Bundle:} $(x_1^*, x_2^*) = (4, 16)$.

\medskip\noindent\rule{\linewidth}{0.4pt}\medskip

\subsubsection*{Exam-Style Problem 11: Perfect Complements Optimization}

\textbf{Problem (Adapted from PS2 Q1(c)):}

Suppose $p_1 = 5$, $p_2 = 1$, and $m = 20$. Determine the optimal consumption bundle for the consumer with the utility function $u(x_1, x_2) = \min\{2x_1, 3x_2\}$.

\textbf{Solution:}

\textbf{Strategy:} Since these are perfect complements, the optimal bundle must satisfy the consumption proportion defined by the kink ($2x_1 = 3x_2$) and the budget constraint.

\textbf{Step 1: Establish the Fixed Consumption Ratio}
\[ 2x_1 = 3x_2 \implies x_2 = \frac{2}{3} x_1 \]

\textbf{Step 2: Apply the Budget Constraint}
The budget line is $p_1 x_1 + p_2 x_2 = m$:
\[ 5x_1 + 1x_2 = 20 \]

\textbf{Step 3: Substitute the Ratio into the Budget Equation and Solve for $x_1^*$}
\[ 5x_1 + 1\left(\frac{2}{3} x_1\right) = 20 \]
Find a common denominator:
\[ \frac{15x_1}{3} + \frac{2x_1}{3} = 20 \]
\[ \frac{17x_1}{3} = 20 \]
\[ x_1^* = \frac{60}{17} \]

\textbf{Step 4: Solve for $x_2^*$}
\[ x_2^* = \frac{2}{3} x_1^* = \frac{2}{3} \cdot \frac{60}{17} = \frac{2 \cdot 20}{17} = \frac{40}{17} \]

\textbf{Optimal Bundle:} $(x_1^*, x_2^*) = \left(\frac{60}{17}, \frac{40}{17}\right)$.

\hrule

\section{Revealed Preference}

The theory of \textbf{Revealed Preference} reverses the logic of consumer choice: instead of using preferences to predict demand, we use observed demand to infer underlying preferences,.

\subsection{Weak Axiom of Revealed Preference (WARP)}

WARP is the fundamental consistency test that observed choices must satisfy if they are generated by a consumer maximizing a stable preference relation,.

\subsubsection*{Clear Definitions of Key Concepts}
\begin{itemize}
    \item \textbf{Directly Revealed Preferred ($x \succ_D y$):} A bundle $x = (x_1, x_2)$ is directly revealed preferred to a bundle $y = (y_1, y_2)$ if $x$ is chosen at prices $(p_1, p_2)$ when $y$ is also affordable at those same prices,.
    \begin{itemize}
        \item \textbf{Condition:} The expenditure on the chosen bundle must be greater than or equal to the cost of the other bundle: \[p_1 x_1 + p_2 x_2 \geq p_1 y_1 + p_2 y_2\].
    \end{itemize}
    \item \textbf{Weak Axiom of Revealed Preference (WARP):} If bundle $x$ is directly revealed preferred to bundle $y$ ($x \succ_D y$) and $x \neq y$, then $y$ cannot be directly revealed preferred to $x$ ($y \not\succ_D x$),.
\end{itemize}

\subsubsection*{Formal Statement and Consistency (LaTeX)}

WARP states that if we observe $x$ chosen over $y$ in Situation $t$ (prices $p$), and $y$ chosen over $x$ in Situation $s$ (prices $q$), consistency requires that the bundle that was rejected in Situation $t$ must not have been affordable in Situation $s$.

\[\text{If } p_1 x_1 + p_2 x_2 \geq p_1 y_1 + p_2 y_2 \text{ and } x \neq y \text{, then it must be that:}\]
\[q_1 y_1 + q_2 y_2 < q_1 x_1 + q_2 x_2\],.

\begin{itemize}
    \item \textbf{Implication:} WARP is a \textbf{necessary condition} for utility maximization. If a consumer violates WARP, their behavior cannot be rationalized by a complete and transitive preference relation that remains stable between the two observations,,.
\end{itemize}

\subsubsection*{$\star$ Exam Focus: Shortcuts \& Mistakes}
\begin{itemize}
    \item \textbf{WARP Failure:} A violation occurs when both bundles are affordable at the budget set where the other was chosen,.
    \item \textbf{Shortcut for No Comparison:} If $p^t \cdot x^t < p^t \cdot x^s$ (i.e., bundle $x^s$ was too expensive when $x^t$ was chosen), then $x^t$ is \textit{not} revealed preferred to $x^s$. This prevents a WARP revelation from one side, and thus, no violation is possible based on that pair of observations,.
    \item \textbf{WARP vs. SARP:} WARP only checks direct comparisons. It may be satisfied even if the agent is not a utility maximizer. The stronger condition, SARP (Strong Axiom of Revealed Preference), is necessary and sufficient to guarantee the existence of a utility function,.
\end{itemize}

\hrulefill\vspace{2ex}

\subsubsection*{Step-by-Step Worked Example: Checking Affordability}

Suppose a consumer chooses $x^A=(4, 4)$ at prices $p^A=(10, 5)$, and $x^B=(2, 6)$ at prices $p^B=(8, 10)$. Check if $x^A$ is revealed preferred to $x^B$.

\textbf{Step 1: Calculate Income $m^A$ (Budget where $x^A$ was chosen)}
\[ m^A = p_1^A x_1^A + p_2^A x_2^A = 10(4) + 5(4) = 40 + 20 = 60 \]

\textbf{Step 2: Check Affordability of $x^B$ at $p^A$}
Calculate the cost of $x^B$ at prices $p^A$:
\[ \text{Cost}(x^B) = p_1^A x_1^B + p_2^A x_2^B = 10(2) + 5(6) = 20 + 30 = 50 \]

\textbf{Step 3: Determine Revealed Preference}
Since $x^B$ cost $50, which is less than m^A=60$, bundle $x^B$ was affordable when $x^A$ was chosen.
\[\text{Cost}(x^B) = 50 < m^A = 60\]
Therefore, $x^A$ is directly revealed preferred to $x^B$: $x^A \succ_D x^B$.

\hrulefill\vspace{2ex}

\subsubsection*{Exam-Style Problem 12: WARP Violation}

\textbf{Problem (Adapted from Practice Midterm Q4):}

We observe the following choices by a consumer over apples ($x_1$) and bananas ($x_2$):
\begin{enumerate}
    \item \textbf{Situation A:} Prices $p^A = (2, 1)$, chosen bundle $x^A = (10, 5)$.
    \item \textbf{Situation B:} Prices $p^B = (2, 2)$, chosen bundle $x^B = (6, 10)$.
\end{enumerate}

Do these choices satisfy the Weak Axiom of Revealed Preference (WARP)?

\textbf{Solution Strategy:} Check if $x^A$ is revealed preferred to $x^B$ AND if $x^B$ is revealed preferred to $x^A$. A violation occurs if both are true.

\textbf{1. Check for $x^A \succ_D x^B$ (Check Affordability in Situation A)}
\begin{itemize}
    \item \textbf{Income in A:} $m^A = 2(10) + 1(5) = 25$.
    \item \textbf{Cost of $x^B$ in A:} $\text{Cost}^A(x^B) = 2(6) + 1(10) = 12 + 10 = 22$.
    \item \textbf{Conclusion 1:} Since $22 < 25$, $x^B$ was affordable when $x^A$ was chosen. Thus, $x^A$ is strictly revealed preferred to $x^B$.
\end{itemize}

\textbf{2. Check for $x^B \succ_D x^A$ (Check Affordability in Situation B)}
\begin{itemize}
    \item \textbf{Income in B:} $m^B = 2(6) + 2(10) = 12 + 20 = 32$.
    \item \textbf{Cost of $x^A$ in B:} $\text{Cost}^B(x^A) = 2(10) + 2(5) = 20 + 10 = 30$.
    \item \textbf{Conclusion 2:} Since $30 < 32$, $x^A$ was affordable when $x^B$ was chosen. Thus, $x^B$ is strictly revealed preferred to $x^A$.
\end{itemize}

\textbf{Final Conclusion:} Since we have revealed that $x^A \succ_D x^B$ and $x^B \succ_D x^A$, the choices \textbf{violate WARP}. The consumer's observed behavior is inconsistent with maximizing a stable preference ordering.

\hrulefill

\section*{5. Revealed Preference (Continued)}

\subsection*{Applications: Consistency Testing}

The core application of Revealed Preference theory is to establish whether a consumer's observed choices are internally consistent, and thus, whether they could have been generated by a rational agent maximizing a stable utility function.

\subsubsection*{Testing Rationality using WARP}

The Weak Axiom of Revealed Preference (WARP) is the fundamental criterion used to test if consumer behavior is consistent with the standard model of rational optimization.

\begin{itemize}
    \item \textbf{Criterion:} If $x$ is chosen when $y$ is affordable (meaning $x$ is revealed preferred to $y$), then $y$ must never be chosen when $x$ is affordable.
    \item \textbf{Purpose:} If observed choices violate WARP, the consumer's behavior cannot be explained by maximizing any stable, rational (complete and transitive) preference relation.
    \item \textbf{Formula (WARP Violation):} A violation occurs if we observe:
    \[\text{If } p \cdot x \geq p \cdot y \text{ and } q \cdot y \geq q \cdot x\]
    where $x$ is chosen at prices $p$, and $y$ is chosen at prices $q$.
\end{itemize}

\subsubsection*{$\star$ Exam Focus: Shortcuts \& Mistakes}

\begin{itemize}
    \item \textbf{Key Distinction (WARP vs. SARP):} WARP is a \textbf{necessary} condition for utility maximization. If WARP is violated, the choices are definitely irrational. However, WARP satisfaction alone is \textbf{not sufficient} to guarantee a utility function exists (SARP is the necessary and sufficient condition for utility maximization).
    \item \textbf{Shortcut: Checking Affordability:} The only way bundle $x$ can be revealed preferred to $y$ is if the expenditure on $x$ at prices $p$ is greater than or equal to the expenditure required to purchase $y$ at the same prices $p$. If $p \cdot x < p \cdot y$, no preference relation is revealed between $x$ and $y$ in that observation.
\end{itemize}

\hrulefill

\subsubsection*{Step-by-Step Application Procedure (WARP Check)}

Given two observations, $(x^A, p^A)$ and $(x^B, p^B)$:

\begin{enumerate}
    \item \textbf{Check Affordability of $x^B$ in Situation A ($x^A \succ_D x^B$):}
    \begin{itemize}
        \item Calculate cost of $x^A$ at $p^A$: $m^A = p^A \cdot x^A$.
        \item Calculate cost of $x^B$ at $p^A$: $\text{Cost}^A(x^B) = p^A \cdot x^B$.
        \item If $m^A \geq \text{Cost}^A(x^B)$, then $x^A$ is revealed preferred to $x^B$.
    \end{itemize}
    \item \textbf{Check Affordability of $x^A$ in Situation B ($x^B \succ_D x^A$):}
    \begin{itemize}
        \item Calculate cost of $x^B$ at $p^B$: $m^B = p^B \cdot x^B$.
        \item Calculate cost of $x^A$ at $p^B$: $\text{Cost}^B(x^A) = p^B \cdot x^A$.
        \item If $m^B \geq \text{Cost}^B(x^A)$, then $x^B$ is revealed preferred to $x^A$.
    \end{itemize}
    \item \textbf{Conclusion:} If both steps 1 and 2 result in a revealed preference, WARP is violated.
\end{enumerate}

\hrulefill

\subsubsection*{Exam-Style Problem 13: Testing WARP Consistency}

\textbf{Problem (Adapted from PS2 Q3(a)):}

A consumer's choices across two periods are observed:

\begin{enumerate}
    \item \textbf{Observation 1 (Period A):} Chosen bundle $x^A = (10, 10)$ at prices $p^A = (5, 7)$.
    \item \textbf{Observation 2 (Period B):} Chosen bundle $x^B = (20, 5)$ at prices $p^B = (5, 4)$.
\end{enumerate}

Determine if these observed choices satisfy the Weak Axiom of Revealed Preference (WARP).

\textbf{Solution:}

\textbf{Step 1: Analyze Affordability in Period A (Checking $x^A \succ_D x^B$):}
\begin{enumerate}
    \item \textbf{Income in A:} $m^A = p^A \cdot x^A = 5(10) + 7(10) = 50 + 70 = 120$.
    \item \textbf{Cost of $x^B$ at $p^A$:} $\text{Cost}^A(x^B) = 5(20) + 7(5) = 100 + 35 = 135$.
    \item \textbf{Comparison:} Since $120 < 135$, bundle $x^B$ was \textbf{not affordable} when $x^A$ was chosen.
    \item \textbf{Conclusion 1:} $x^A$ is \textbf{NOT} revealed preferred to $x^B$. (No contradiction yet).
\end{enumerate}

\textbf{Step 2: Analyze Affordability in Period B (Checking $x^B \succ_D x^A$):}
\begin{enumerate}
    \item \textbf{Income in B:} $m^B = p^B \cdot x^B = 5(20) + 4(5) = 100 + 20 = 120$.
    \item \textbf{Cost of $x^A$ at $p^B$:} $\text{Cost}^B(x^A) = 5(10) + 4(10) = 50 + 40 = 90$.
    \item \textbf{Comparison:} Since $120 \geq 90$, bundle $x^A$ \textbf{was affordable} when $x^B$ was chosen.
    \item \textbf{Conclusion 2:} $x^B$ is strictly revealed preferred to $x^A$ ($x^B \succ_D x^A$).
\end{enumerate}

\textbf{Final Conclusion:} Since $x^A$ is not revealed preferred to $x^B$ (Step 1), and $x^B$ is revealed preferred to $x^A$ (Step 2), there is no symmetric contradiction. The choices \textbf{satisfy WARP}. The observed behavior is consistent with the maximization of a rational preference relation.

\hrulefill

\section*{Demand}
\subsection*{Individual Demand}

The \textbf{Individual Demand Function} summarizes the solution to the consumer's utility maximization problem, specifying the optimal quantities chosen for a given set of prices and income.

\vspace{1em}
\noindent\textbf{Clear Definitions of Key Concepts}
\begin{itemize}
    \item \textbf{Consumer's Problem:} The fundamental objective is to choose the consumption bundle that provides the highest utility while satisfying the budget constraint. \[\max_{x_1, x_2} u(x_1, x_2) \quad \text{subject to} \quad p_1 x_1 + p_2 x_2 \leq m.\]
    \item \textbf{Demanded Bundle:} The optimal choice of goods 1 and 2 at a given set of prices and income.
    \item \textbf{Demand Function ($x_i$):} A function that relates the optimal choice of good $i$ to the different values of prices ($p_1, p_2$) and income ($m$). \[x_i^* = x_i(p_1, p_2, m).\]
    \item \textbf{Inverse Demand Function ($p_i(x_i)$):} A function that measures the price at which a given quantity will be demanded. It measures the consumer's marginal willingness to pay for an additional unit of the good at that consumption level.
\end{itemize}

\vspace{1em}
\noindent\textbf{Relevant Formulas and Derivations (LaTeX)}
\vspace{0.5em}

\noindent The demand functions are typically derived by solving a system of two equations:
\begin{enumerate}
    \item \textbf{Tangency Condition (Interior Optimum):} \[\frac{\text{MU}_1}{\text{MU}_2} = \frac{p_1}{p_2} \iff \frac{\partial u(x_1, x_2)/\partial x_1}{\partial u(x_1, x_2)/\partial x_2} = \frac{p_1}{p_2}.\]
    \item \textbf{Budget Constraint (Binding):} \[p_1 x_1 + p_2 x_2 = m.\]
\end{enumerate}

\vspace{1em}
\noindent\textbf{Example: General Cobb-Douglas Demand Derivation}
\vspace{0.5em}

\noindent For the utility function $u(x_1, x_2) = x_1^c x_2^d$, or its monotonic transformation $v(x_1, x_2) = c \ln x_1 + d \ln x_2$:
\begin{enumerate}
    \item \textbf{Calculate MRS:} \[ \text{MRS} = -\frac{c/x_1}{d/x_2} = -\frac{c x_2}{d x_1}. \]
    \item \textbf{Apply Tangency Condition:} \[ \frac{c x_2}{d x_1} = \frac{p_1}{p_2} \implies x_2 = \frac{d p_1}{c p_2} x_1. \]
    \item \textbf{Substitute into Budget Constraint:} 
    \begin{align*} 
    p_1 x_1 + p_2 \left(\frac{d p_1}{c p_2} x_1\right) &= m \\ 
    p_1 x_1 + \frac{d p_1}{c} x_1 &= m \\ 
    p_1 x_1 \left(1 + \frac{d}{c}\right) &= m \implies p_1 x_1 \left(\frac{c+d}{c}\right) = m 
    \end{align*}
    \item \textbf{Solve for $x_1^*$ and $x_2^*$ (Demand Functions):} \[ x_1^*(p_1, p_2, m) = \frac{c}{c+d} \frac{m}{p_1} \qquad x_2^*(p_1, p_2, m) = \frac{d}{c+d} \frac{m}{p_2}. \]
\end{enumerate}

\vspace{0.5em}
\hrulefill
\vspace{0.5em}

\noindent\textbf{$\star$ Exam Focus: Shortcuts \& Mistakes}
\begin{itemize}
    \item \textbf{Cobb-Douglas Shortcut:} For $u(x_1, x_2) = x_1^c x_2^d$, the fraction of income spent on good 1 is constant and equal to $s_1 = \frac{c}{c+d}$. The demand for good 1 is simply $s_1 \cdot \frac{m}{p_1}$.
    \item \textbf{Monotonicity Check:} The first step in optimization is confirming monotonicity. If preferences are strictly monotonic, the budget constraint must be binding ($\leq m$ becomes $= m$).
    \item \textbf{Concave/Substitutes Trap:} If preferences are concave or perfect substitutes, the tangency condition is likely to fail (concave preferences) or provide an interior point that is not the optimum (substitutes). Always use the appropriate method (corner solutions for concave/substitutes, kink points for complements).
\end{itemize}

\vspace{0.5em}
\hrulefill
\vspace{0.5em}

\noindent\textbf{Exam-Style Problem 14: Deriving and Using Cobb-Douglas Demand}
\vspace{0.5em}

\noindent A consumer has the utility function $u(x_1, x_2) = x_1^3 x_2^2$.
\begin{enumerate}
    \item Derive the demand function for $x_1$ and $x_2$.
    \item Using the derived function, find the optimal bundle when $p_1=1$, $p_2=2$, and $m=10$.
\end{enumerate}

\vspace{1em}
\noindent\textbf{Solution:}
\vspace{0.5em}

\noindent\textbf{Part 1: Deriving the Demand Function (General Solution)}
\vspace{0.5em}

\noindent This is a Cobb-Douglas function with $c=3$ and $d=2$. $c+d = 5$.
\begin{enumerate}
    \item \textbf{Apply Cobb-Douglas Demand Formula (Shortcut):} Fraction spent on $x_1$: $\frac{c}{c+d} = \frac{3}{5}$. Fraction spent on $x_2$: $\frac{d}{c+d} = \frac{2}{5}$.
    \item \textbf{Demand Functions (LaTeX):} \[ x_1^*(p_1, p_2, m) = \frac{3}{5} \frac{m}{p_1} \qquad x_2^*(p_1, p_2, m) = \frac{2}{5} \frac{m}{p_2} \]
\end{enumerate}

\vspace{0.5em}
\noindent\textbf{Part 2: Finding the Optimal Bundle (Specific Values)}
\vspace{0.5em}

\noindent Given $p_1=1$, $p_2=2$, and $m=10$.
\begin{enumerate}
    \item \textbf{Calculate $x_1^*$:} \[ x_1^* = \frac{3}{5} \frac{10}{1} = 6 \]
    \item \textbf{Calculate $x_2^*$:} \[ x_2^* = \frac{2}{5} \frac{10}{2} = \frac{20}{10} = 2 \]
\end{enumerate}

\noindent\textbf{Optimal Bundle:} $(x_1^*, x_2^*) = (6, 2)$.

\vspace{0.5em}
\noindent\textit{(Verification: Check Budget Constraint: $1(6) + 2(2) = 6 + 4 = 10$. Satisfied.)}

\vspace{1em}
\hrulefill

\section{6. Demand (Continued)}
\subsection{(b) Comparative Statics}

\textbf{Comparative Statics} is the study of how the consumer's optimal choice (demand) changes in response to changes in economic parameters (prices and income).

When the price of a good changes, the change in quantity demanded ($\Delta x_i$) is decomposed into two distinct effects: the \textbf{Substitution Effect (SE)} and the \textbf{Income Effect (IE)}.

\subsubsection*{Clear Definitions of Key Concepts}

\begin{tabular*}{\textwidth}{@{\extracolsep{\fill}} l p{0.3\textwidth} p{0.4\textwidth} l}
\toprule
\textbf{Concept} & \textbf{Definition} & \textbf{Implication for Demand} & \textbf{Source} \\
\midrule
\textbf{Substitution Effect (SE)} & The change in demand resulting solely from the change in the \textit{relative price} of the goods, holding the consumer's purchasing power constant (Slutsky definition). & Always negative: $\Delta x_i^s < 0$. If $p_i \uparrow$, the SE dictates $x_i \downarrow$ (The good is consumed less because it is now relatively more expensive). & \\
\addlinespace
\textbf{Income Effect (IE)} & The change in demand due to the necessary adjustment in the consumer’s purchasing power (real income) caused by the price change, holding relative prices constant. & Depends on the nature of the good (Normal or Inferior). & \\
\addlinespace
\textbf{Normal Good} & A good for which demand increases when income ($m$) increases. & IE moves in the same direction as the price change (i.e., $p_i \uparrow \implies x_i \downarrow$). $\frac{\partial x_i}{\partial m} > 0$. & \\
\addlinespace
\textbf{Inferior Good} & A good for which demand decreases when income ($m$) increases. & IE moves in the opposite direction of the price change (i.e., $p_i \uparrow \implies x_i \uparrow$). $\frac{\partial x_i}{\partial m} < 0$. & \\
\addlinespace
\textbf{Giffen Good} & An extreme inferior good where the IE is so strong that it dominates the SE, causing total demand to rise when price rises (violating the Law of Demand). & Only possible if the good is inferior and $| \text{IE}|$ & \\
\bottomrule
\end{tabular*}


\subsubsection*{Relevant Formulas (Slutsky Decomposition)}

The total change in demand ($\Delta x_i$) is the sum of the substitution effect ($\Delta x_i^s$) and the income effect ($\Delta x_i^n$): \[\Delta x_i = \Delta x_i^s + \Delta x_i^n\]

The core of the Slutsky method involves finding the \textbf{compensated income ($m'$)} needed to make the original bundle $(x_1, x_2)$ affordable at the new prices $(p_1', p_2')$: \[m' = p_1' x_1 + p_2' x_2\]

The relationship between the partial derivatives (calculus approach) is given by the \textbf{Derivative Form of the Slutsky Equation} : \[\frac{\partial x_1(p_1, p_2, m)}{\partial p_1} = \frac{\partial x_1^s}{\partial p_1} - \frac{\partial x_1(p_1, p_2, m)}{\partial m} x_1\] Where:

\begin{itemize}
    \item $\frac{\partial x_1}{\partial p_1}$ is the total effect.
    \item $\frac{\partial x_1^s}{\partial p_1}$ is the substitution effect (which is always negative).
    \item $-\frac{\partial x_1}{\partial m} x_1$ is the income effect.
\end{itemize}

\vspace{5pt}\noindent\rule{\linewidth}{0.4pt}\vspace{5pt}

\subsubsection*{$\star$ Exam Focus: Shortcuts \& Pitfalls}

\begin{itemize}
    \item \textbf{Law of Demand:} If a good is \textbf{Normal} , the Law of Demand must hold ($\frac{\partial x_i}{\partial p_i} < 0$), because the SE and IE both push demand in the same direction.
    \item \textbf{Income Effect Sign:} If the price $p_i$ increases, the real income falls.
    \begin{itemize}
        \item Normal Goods: Income effect causes demand for $x_i$ to fall ($x_i \downarrow$).
        \item Inferior Goods: Income effect causes demand for $x_i$ to rise ($x_i \uparrow$).
    \end{itemize}
    \item \textbf{Slutsky Compensation:} Remember that the compensated income $m'$ is calculated to make the \textit{original bundle} affordable at the \textit{new prices}.
\end{itemize}

\vspace{5pt}\noindent\rule{\linewidth}{0.4pt}\vspace{5pt}

\subsubsection*{Step-by-Step Worked Example: Slutsky Decomposition (Cobb-Douglas)}

Suppose the consumer has the demand functions $x_1^{\_}(p_1, p_2, m) = \frac{m}{2p_1}$ and $x_2^{\_}(p_1, p_2, m) = \frac{m}{2p_2}$ (representing Cobb-Douglas preferences $u(x_1, x_2) = x_1 x_2$). Initial parameters: $p_1=5$, $p_2=1$, $m=20$. The price of good 1 increases to $p_1'=6$. Decompose the change in demand for $x_1$.

\textbf{Step 1: Calculate Initial Optimal Bundle (A)}
\[(x_1^A, x_2^A) = \left(\frac{20}{2(5)}, \frac{20}{2(1)}\right) = (2, 10)\]

\textbf{Step 2: Calculate Final Optimal Bundle (C)}
\[(x_1^C, x_2^C) = \left(\frac{20}{2(6)}, \frac{20}{2(1)}\right) = \left(\frac{10}{6}, 10\right) = \left(\frac{5}{3}, 10\right)\]

\textbf{Step 3: Calculate Total Change in Demand ($\Delta x_1$)}
\[\Delta x_1 = x_1^C - x_1^A = \frac{5}{3} - 2 = \frac{5}{3} - \frac{6}{3} = -\frac{1}{3}\]

\textbf{Step 4: Calculate Compensated Income ($m'$)}
Compensated income needed to buy the original bundle $A=(2, 10)$ at new prices $(6, 1)$:
\[m' = p_1' x_1^A + p_2 x_2^A = 6(2) + 1(10) = 12 + 10 = 22\]

\textbf{Step 5: Calculate Compensated Demand ($x_1^s$)}
The compensated demand is calculated using the new price $p_1'=6$ and the compensated income $m'=22$:
\[x_1^s = x_1(6, 1, 22) = \frac{22}{2(6)} = \frac{22}{12} = \frac{11}{6}\]

\textbf{Step 6: Calculate Substitution Effect (SE)}
\[\Delta x_1^s = x_1^s - x_1^A = \frac{11}{6} - 2 = \frac{11}{6} - \frac{12}{6} = -\frac{1}{6}\]

\textbf{Step 7: Calculate Income Effect (IE)}
\[\Delta x_1^n = x_1^C - x_1^s = \frac{5}{3} - \frac{11}{6} = \frac{10}{6} - \frac{11}{6} = -\frac{1}{6}\]

\textbf{Step 8: Verification}
\[\Delta x_1 = \Delta x_1^s + \Delta x_1^n = -\frac{1}{6} + \left(-\frac{1}{6}\right) = -\frac{2}{6} = -\frac{1}{3}\]
The decomposition is correct. (Since the IE is negative, $x_1$ is a Normal Good).

\vspace{5pt}\noindent\rule{\linewidth}{0.4pt}\vspace{5pt}

\subsubsection*{Exam-Style Problem 15: Identifying Good Types}

\textbf{Problem (Based on Price Elasticity and Income Elasticity Concepts):}

A consumer has the demand function $x_1^*(p_1, p_2, m) = \frac{p_2 m}{p_1(p_1 + p_2)}$.
\begin{enumerate}
    \item Is good 1 a normal good or an inferior good?
    \item If the price of good 2 ($p_2$) increases, does the demand curve for good 1 shift left or right? Classify the relationship between good 1 and good 2.
\end{enumerate}

\textbf{Solution:}

\textbf{Part 1: Normal or Inferior Good?}

\textbf{Strategy:} Analyze the partial derivative of $x_1^*$ with respect to income $m$.
\begin{enumerate}
    \item \textbf{Calculate $\partial x_1 / \partial m$:}
    \[ \frac{\partial x_1}{\partial m} = \frac{p_2}{p_1(p_1 + p_2)} \]
    \item \textbf{Determine Sign:} Since prices ($p_1, p_2$) and quantities are assumed non-negative, the numerator ($p_2$) and the denominator ($p_1(p_1 + p_2)$) are strictly positive.
    \item \textbf{Conclusion:} $\frac{\partial x_1}{\partial m} > 0$. Good 1 is a \textbf{Normal Good}.
\end{enumerate}

\textbf{Part 2: Relationship between Good 1 and Good 2 (Substitutes or Complements)}

\textbf{Strategy:} Analyze the partial derivative of $x_1^*$ with respect to the price of the other good, $p_2$.
\begin{enumerate}
    \item \textbf{Calculate $\partial x_1 / \partial p_2$ (using the quotient rule or treating $p_1 m$ as constants):}
    \[ x_1 = \frac{m p_2}{p_1^2 + p_1 p_2} \]
    \[ \frac{\partial x_1}{\partial p_2} = \frac{m p_1^2}{(p_1(p_1 + p_2))^2} \]
    \textit{(Self-Correction/Reference: The source solution for a similar function yields a simplified positive result. Using the provided function and simplifying the derivative leads to:}
    \[ \frac{\partial x_1}{\partial p_2} = \frac{m p_1}{(p_1+p_2)^2} > 0 \]
    \item \textbf{Determine Sign and Shift:} Since $\frac{\partial x_1}{\partial p_2} > 0$, an increase in the price of good 2 ($p_2 \uparrow$) causes the quantity demanded of good 1 ($x_1$) to increase at any given price $p_1$.
    \item \textbf{Conclusion:} The demand curve for good 1 shifts to the \textbf{right}. Goods 1 and 2 are (Gross) \textbf{Substitutes} (since $x_1$ increases when $p_2$ increases).
\end{enumerate}

\hrulefill

\section*{6. Demand (Continued)}
\subsection*{(c) Market Demand}

\textbf{Market Demand} (or \textbf{Aggregate Demand}) is the total quantity of a good demanded by all consumers in the market at a given set of prices and income distribution. It is derived by summing the quantities demanded by all individuals.

\subsubsection*{Clear Definitions and Formulas}
\begin{itemize}
    \item \textbf{Market Demand Function ($X_i$):} The sum of individual demand functions ($x_{i}^j$) for all $n$ consumers. 
    \[\text{Aggregate Demand: } X_i(p_1, p_2, m_1, \dots, m_n) = \sum_{j=1}^{n} x_{i}^j(p_1, p_2, m_j)\]
    \item \textbf{Horizontal Summation:} Market demand curves are found by summing the quantities demanded (the horizontal axis values) at every given price.
    \item \textbf{Market Demand Dependence:} Aggregate demand depends on prices ($p_1, p_2$) and the distribution of income among consumers ($m_1, \dots, m_n$). If the "representative consumer" assumption is used, it is simplified to $X_i(p_1, p_2, M)$, where $M = \sum m_j$ is aggregate income.
\end{itemize}

\subsubsection*{$\star$ Exam Focus: Shortcuts \& Mistakes}
\begin{itemize}
    \item \textbf{Process Priority:} Always sum \textbf{quantities} (horizontally), not prices (vertically), to find the market demand curve.
    \item \textbf{Kink Identification:} When summing individual "linear" demand functions, remember that individual demand cannot be negative. You must include $\max{D_i(p), 0}$ for each individual demand. Kinks occur in the aggregate curve at prices where one or more individual demands drop to zero.
    \item \textbf{Shifts:} If good $i$ is a normal good for an individual, an increase in that individual's income shifts the aggregate demand curve outward. If goods $i$ and $k$ are substitutes, increasing $p_k$ shifts $X_i$ outward; if they are complements, increasing $p_k$ shifts $X_i$ inward.
\end{itemize}

\subsubsection*{Step-by-Step Worked Example: Horizontal Summation (Discrete Good)}

Suppose the market consists of two consumers choosing a discrete good.
\begin{itemize}
    \item Consumer 1 has a reservation price $r_1 = 15$.
    \item Consumer 2 has a reservation price $r_2 = 10$.
\end{itemize}
Derive the market demand curve, $X(p)$.

\textbf{Step 1: Define Individual Demand Correspondence ($x_i(p)$)}
\begin{itemize}
    \item Consumer 1 buys 1 unit if $p \leq 15$, and 0 otherwise. 
    \[ x_1(p) = \begin{cases} 1 & \text{if } p \leq 15 \\ 0 & \text{if } p > 15 \end{cases} \]
    \item Consumer 2 buys 1 unit if $p \leq 10$, and 0 otherwise. 
    \[ x_2(p) = \begin{cases} 1 & \text{if } p \leq 10 \\ 0 & \text{if } p > 10 \end{cases} \]
\end{itemize}

\textbf{Step 2: Sum Demands across all relevant price ranges} 

We establish price intervals based on the reservation prices:
\begin{enumerate}
    \item \textbf{$p > 15$:} Neither consumer buys. 
    \[ X(p) = x_1(p) + x_2(p) = 0 + 0 = 0 \]
    \item \textbf{$10 < p \leq 15$:} Only Consumer 1 buys (since $p \leq 15$ but $p > 10$). 
    \[ X(p) = 1 + 0 = 1 \]
    \item \textbf{$0 \leq p \leq 10$:} Both consumers buy. 
    \[ X(p) = 1 + 1 = 2 \]
\end{enumerate}

\textbf{Step 3: Define Market Demand Curve}
\[ X(p) = \begin{cases} 2 & \text{if } 0 \leq p \leq 10 \\ 1 & \text{if } 10 < p \leq 15 \\ 0 & \text{if } p > 15 \end{cases} \]
(This yields a staircase shape, typical for market demand of discrete goods).

\bigskip
\noindent\rule{\linewidth}{0.4pt}
\bigskip

\subsubsection*{Exam-Style Problem 16: Horizontal Summation of Linear Demand}

The market consists of two consumers with the following demand functions for good 1: 
\begin{align*}
x_A(p_1) &= 10 - p_1 \\
x_B(p_1) &= 20 - 4p_1
\end{align*}
Assuming $x_i(p_1) = \max{D_i(p_1), 0}$, derive the market demand function $X(p_1)$.

\textbf{Solution Strategy:} Find the shut-off price for each consumer, then define the total market quantity $X(p_1) = x_A(p_1) + x_B(p_1)$ across the defined price ranges.

\textbf{Step 1: Determine the Shut-off Price for each Consumer}
\begin{enumerate}
    \item Consumer A stops demanding when $x_A(p_1) = 0$: 
    \[ 10 - p_1 = 0 \implies p_1 = 10 \]
    \item Consumer B stops demanding when $x_B(p_1) = 0$: 
    \[ 20 - 4p_1 = 0 \implies p_1 = 5 \]
\end{enumerate}

\textbf{Step 2: Define Price Ranges and Summation} 

The critical prices are $p_1=5$ and $p_1=10$.
\begin{enumerate}
    \item \textbf{High Price Range ($p_1 > 10$):} Neither consumer demands the good. 
    \[ X(p_1) = 0 + 0 = 0 \]
    \item \textbf{Middle Price Range ($5 < p_1 \leq 10$):} Consumer A demands, but Consumer B does not. 
    \[ X(p_1) = (10 - p_1) + 0 = 10 - p_1 \]
    \item \textbf{Low Price Range ($0 \leq p_1 \leq 5$):} Both consumers demand the good. 
    \[ X(p_1) = (10 - p_1) + (20 - 4p_1) = 30 - 5p_1 \]
\end{enumerate}

\textbf{Step 3: Write the Final Market Demand Function (Correspondence)}
\[ X(p_1) = \begin{cases} 30 - 5p_1 & \text{if } 0 \leq p_1 \leq 5 \\ 10 - p_1 & \text{if } 5 < p_1 \leq 10 \\ 0 & \text{if } p_1 > 10 \end{cases} \]
(Note the kink at $p_1=5$).

\hrulefill

\section*{Income and Substitution Effects}
\subsection*{Hicksian Decomposition (Slutsky Method for Calculation)}

When the price of a good changes, the \textbf{Total Effect} on demand is decomposed into the \textbf{Substitution Effect (SE)}, caused by the change in relative prices, and the \textbf{Income Effect (IE)}, caused by the change in purchasing power.

The Hicksian decomposition seeks to find the substitution effect by conceptually adjusting income so the consumer remains on the original indifference curve (constant utility). In practice, especially for calculations required in this course, we often use the \textbf{Slutsky definition} for decomposition, which adjusts income to keep the \textit{original bundle affordable}. For infinitesimal changes (calculus approach), the Slutsky and Hicksian substitution effects are equivalent.

\subsubsection*{Clear Definitions of Key Concepts (Slutsky Framework)}
\begin{itemize}
	\item \textbf{Total Change in Demand ($\Delta x_i$):} The difference between the final optimal choice (C) and the initial optimal choice (A). \[\Delta x_i = x_i(p'_1, p'_2, m) - x_i(p_1, p_2, m)\]
	\item \textbf{Slutsky Compensated Income ($m^S$):} The hypothetical income required to make the consumer's original consumption bundle ($x_A = (x_1^A, x_2^A)$) exactly affordable at the new prices ($p'$). This isolates the SE by holding purchasing power constant. \[m^S = p'_1 x_1^A + p'_2 x_2^A\]
	\item \textbf{Substitution Effect (SE):} The change in demand resulting from changing relative prices from $p$ to $p'$, while compensating the consumer with $m^S$ (making the original bundle affordable). This represents the movement from the original bundle (A) to the compensated bundle ($x^S$). \[\text{SE} (\Delta x_i^s) = x_i(p'_1, p'_2, m^S) - x_i(p_1, p_2, m)\]
	\item \textbf{Income Effect (IE):} The change in demand resulting from adjusting income from the compensated level ($m^S$) back to the actual level ($m$), holding relative prices fixed at $p'$. This represents the movement from the compensated bundle ($x^S$) to the final bundle (C). \[\text{IE} (\Delta x_i^n) = x_i(p'_1, p'_2, m) - x_i(p'_1, p'_2, m^S)\]
\end{itemize}

\subsubsection*{Relevant Formulas (Slutsky Identity)}
The total change in demand is always the sum of the two effects: \[\underbrace{x_i(p'_1, p'_2, m) - x_i(p_1, p_2, m)}_{\text{Total Change}} = \underbrace{\left[x_i(p'_1, p'_2, m^S) - x_i(p_1, p_2, m)\right]}_{\text{Substitution Effect}} + \underbrace{\left[x_i(p'_1, p'_2, m) - x_i(p'_1, p'_2, m^S)\right]}_{\text{Income Effect}}\]

\subsubsection*{$\star$ Exam Focus: Shortcuts \& Pitfalls}
\begin{itemize}
	\item \textbf{Sign of SE:} The substitution effect is \textbf{always negative}. If $p_1$ increases, $x_1^s$ must decrease (and vice versa).
	\item \textbf{Sign of IE:} For a Normal Good, the IE reinforces the SE. For an Inferior Good, the IE opposes the SE.
	\item \textbf{Giffen Good:} Requires the good to be Inferior AND the IE to be stronger than the SE ($|\text{IE}| > |\text{SE}|$).
\end{itemize}

\subsubsection*{Step-by-Step Problem Solving Strategy (Slutsky)}
\begin{enumerate}
	\item \textbf{Calculate Initial Demand (A):} Find $(x_1^A, x_2^A)$ using original prices ($p_1, p_2$) and income ($m$).
	\item \textbf{Calculate Final Demand (C):} Find $(x_1^C, x_2^C)$ using new prices ($p'_1, p'_2$) and income ($m$).
	\item \textbf{Calculate Total Change:} $\Delta x_i = x_i^C - x_i^A$.
	\item \textbf{Calculate Compensated Income ($m^S$):} Use the formula $m^S = p'_1 x_1^A + p'_2 x_2^A$.
	\item \textbf{Calculate Compensated Demand ($x^S$):} Find $(x_1^S, x_2^S)$ using new prices ($p'_1, p'_2$) and $m^S$.
	\item \textbf{Calculate SE:} $\Delta x_i^s = x_i^S - x_i^A$. (Must be $\leq 0$ for the good whose price changed).
	\item \textbf{Calculate IE:} $\Delta x_i^n = x_i^C - x_i^S$. (Verification: $\Delta x_i^s + \Delta x_i^n$ must equal $\Delta x_i$).
\end{enumerate}
\hrulefill

\subsection*{Exam-Style Problem 17: Slutsky Decomposition}

\textbf{Problem (Adapted from PS2 Q5(a)):}

A consumer has the following demand functions for good 1 and good 2 (derived from Cobb-Douglas preferences $u(x_1, x_2) = x_1 x_2$): \[x_1(p_1, p_2, m) = \frac{m}{2p_1} \quad \text{and} \quad x_2(p_1, p_2, m) = \frac{m}{2p_2}\] Initial parameters are $p_1=5$, $p_2=1$, and $m=20$. The price of good 1 increases to $p_1'=6$. Decompose the total change in demand for both goods into the substitution and income effects.

\textbf{Solution:}

\textbf{Step 1: Calculate Initial Optimal Bundle (A)} \[ x_1^A = \frac{20}{2(5)} = 2 \] \[ x_2^A = \frac{20}{2(1)} = 10 \] Initial Bundle: $(2, 10)$.

\textbf{Step 2: Calculate Final Optimal Bundle (C)} New prices $p'=(6, 1)$, $m=20$: \[ x_1^C = \frac{20}{2(6)} = \frac{10}{6} = \frac{5}{3} \] \[ x_2^C = \frac{20}{2(1)} = 10 \] Final Bundle: $(5/3, 10)$.

\textbf{Step 3: Calculate Total Change in Demand ($\Delta x_i$)} \[ \Delta x_1 = x_1^C - x_1^A = \frac{5}{3} - 2 = -\frac{1}{3} \] \[ \Delta x_2 = x_2^C - x_2^A = 10 - 10 = 0 \]

\textbf{Step 4: Calculate Compensated Income ($m^S$)} $m^S$ is the income needed to afford $(2, 10)$ at new prices $(6, 1)$: \[ m^S = 6(2) + 1(10) = 12 + 10 = 22 \]

\textbf{Step 5: Calculate Compensated Demand ($x^S$)} Using $p'=(6, 1)$ and $m^S=22$: \[ x_1^S = \frac{22}{2(6)} = \frac{11}{6} \] \[ x_2^S = \frac{22}{2(1)} = 11 \] Compensated Bundle: $(11/6, 11)$.

\begin{center}
\begin{tabular}{p{7.5cm}ll}
\toprule
Effect& Good 1 (x1)& Good 2 (x2) \\
\midrule
\textbf{Step 6: Substitution Effect (SE)} ($\Delta x_i^s = x^S - x^A$)& $\frac{11}{6} - 2 = -\frac{1}{6}$& $11 - 10 = 1$ \\
\addlinespace
\textbf{Step 7: Income Effect (IE)} ($\Delta x_i^n = x^C - x^S$)& $\frac{5}{3} - \frac{11}{6} = \frac{10}{6} - \frac{11}{6} = -\frac{1}{6}$& $10 - 11 = -1$ \\
\addlinespace
\textbf{Total Change (Verification)} ($\Delta x_i^s + \Delta x_i^n$)& $-\frac{1}{6} + (-\frac{1}{6}) = -\frac{2}{6} = -\frac{1}{3}$& $1 + (-1) = 0$ \\
\bottomrule
\end{tabular}
\end{center}

\textbf{Summary of Decomposition:}
\begin{center}
\begin{tabular}{@{}lccc@{}}
\toprule
 & Substitution Effect & Income Effect & Total Change \\
\midrule
\textbf{Good 1} & -1/6 & -1/6 & -1/3 \\
\textbf{Good 2} & 1 & -1 & 0 \\
\bottomrule
\end{tabular}
\end{center}

\textbf{Interpretation:} Good 1 is a Normal Good (since IE is negative when $p_1$ increases, matching the fall in real income).

\hrulefill

\section{Income and Substitution Effects (Continued)}
\subsection{Graphical Intuition}

The Slutsky decomposition (which is the calculation method used in the sources) and the Hicksian decomposition offer slightly different graphical interpretations of how the price change is split into substitution and income effects, but both fundamentally rely on pivoting and shifting the budget line.

\subsubsection*{The Slutsky Decomposition (Pivot and Shift)}

The Slutsky method decomposes the total movement from the original optimal choice (A) to the final optimal choice (C) into two steps: a pivot (SE) and a parallel shift (IE).

\begin{enumerate}
    \item \textbf{Substitution Effect (SE): The Pivot}
    \begin{itemize}
        \item \textbf{Goal:} Isolate the effect of the change in relative prices ($\frac{p_1'}{p_2}$), holding purchasing power constant in the sense that the \textit{original bundle (A) remains affordable}.
        \item \textbf{Graphical Movement:} The budget line pivots around the initial optimal choice (A) to reflect the new relative price ratio ($-\frac{p_1'}{p_2}$).
        \item \textbf{Result:} The consumer moves from the original bundle (A) to a compensated bundle ($x^S$) on this pivoted budget line. Since preferences are convex, $x^S$ must involve less of the good whose price increased, ensuring the SE is negative.
    \end{itemize}
    \item \textbf{Income Effect (IE): The Shift}
    \begin{itemize}
        \item \textbf{Goal:} Account for the change in real income (purchasing power) necessary to move from the compensated position back to the actual budget line.
        \item \textbf{Graphical Movement:} The compensated budget line (which has slope $-\frac{p_1'}{p_2}$) shifts parallel to itself until it reaches the final budget line.
        \item \textbf{Result:} The consumer moves from the compensated bundle ($x^S$) to the final optimal choice (C). The direction of this shift determines if the good is normal or inferior.
    \end{itemize}
\end{enumerate}

\subsubsection*{The Hicksian Decomposition (Constant Utility)}

The Hicksian approach defines the substitution effect by holding the consumer's utility constant, ensuring the consumer remains on the original indifference curve.

\begin{enumerate}
    \item \textbf{Substitution Effect (SE): Rolling the Budget Line}
    \begin{itemize}
        \item \textbf{Goal:} Find the budget line with the \textit{new price ratio} ($-\frac{p_1'}{p_2}$) that is just tangent to the \textit{original indifference curve}.
        \item \textbf{Graphical Movement:} The budget line "rolls" around the original indifference curve until it achieves the slope of the new relative prices.
        \item \textbf{Result:} The consumer moves from the original bundle (A) to the compensated bundle ($x^H$) on the \textit{original indifference curve}.
    \end{itemize}
    \item \textbf{Income Effect (IE): Parallel Shift}
    \begin{itemize}
        \item \textbf{Goal:} Restore the utility change by shifting from the Hicksian compensated budget line to the final actual budget line.
    \end{itemize}
\end{enumerate}

\textit{Note: While the Hicksian approach is conceptually superior for defining the SE (constant utility), the Slutsky approach is more commonly used in applied calculations because it requires only observable demand functions and budget parameters, rather than knowledge of the utility level/indifference curve.}

\vspace{1em}
\textbf{$\star$ Exam Focus: Shortcuts \& Pitfalls}

\begin{itemize}
    \item \textbf{Pitfall: Graphical Equivalence:} Do not confuse the Slutsky compensated bundle ($x^S$) with the Hicksian compensated bundle ($x^H$). $x^S$ is guaranteed to be affordable at the new prices, while $x^H$ is guaranteed to yield the original utility level.
    \item \textbf{IE Direction Check:} In the graph, if the IE is a shift toward the origin (i.e., less of the good is demanded as income falls/purchasing power decreases), the good is \textbf{Normal}. If the IE is a shift away from the origin (i.e., more of the good is demanded as income falls), the good is \textbf{Inferior}.
\end{itemize}

\hrulefill

\subsubsection*{Exam-Style Problem 18: Perfect Complements and Price Change}

\textbf{Problem (Graphical/Conceptual Check):}

Suppose the consumer has perfect complements preferences, $u(x_1, x_2) = \min\{x_1, x_2\}$. The consumer is initially at point A. The price of good 1 increases. Graphically illustrate the Slutsky decomposition (SE and IE) for the change in demand for $x_1$.

\vspace{1em}
\textbf{Solution:}

\vspace{1em}
\textbf{Strategy:} Perfect complements have L-shaped indifference curves, where the optimal bundle is always at the kink. This preference type implies a unique outcome for the Substitution Effect under the Slutsky method.

\begin{enumerate}
    \item \textbf{Initial State (A):} The optimal point A lies on the kink of the original budget line and the highest indifference curve.
    \item \textbf{Final State (C):} When $p_1$ increases, the final budget line pivots inward, and the new optimal point C occurs at the new tangency (kink).
    \item \textbf{Slutsky Compensation ($m^S$):} The compensated income $m^S$ makes the original bundle A affordable at the new steeper price ratio. The compensated budget line pivots around A.
    \item \textbf{Substitution Effect (SE):} The SE is the movement from A to the optimal choice on the compensated budget line. Since the optimal choice must occur at a kink (where $x_1=x_2$), and the compensated budget line passes through A, the point that maximizes utility on the compensated line \textit{is still A}.
    \begin{itemize}
        \item Therefore, the \textbf{Substitution Effect is zero} ($\Delta x_1^s = 0$).
    \end{itemize}
    \item \textbf{Income Effect (IE):} The IE is the remaining movement from $x^S=A$ to the final choice C.
    \begin{itemize}
        \item Therefore, the \textbf{Total Effect equals the Income Effect} ($\Delta x_1 = \Delta x_1^n$).
    \end{itemize}
\end{enumerate}

\textbf{Graphical Illustration (Key Features):}
\begin{itemize}
    \item Indifference curves are L-shaped.
    \item The original optimum (A) is the kink where the original budget line hits the highest IC.
    \item The pivoted budget line for the SE passes through A.
    \item The compensated bundle $x^S$ equals $A$.
    \item The final bundle (C) is the new optimal kink point.
    \item The entire movement from A to C is labeled as the Income Effect.
\end{itemize}

\hrulefill

\section{Income and Substitution Effects (Continued)}

\subsection{Special Cases: Giffen Goods}

A \textbf{Giffen Good} is an extreme theoretical curiosity where demand violates the \textbf{Law of Demand} , meaning consumption increases when its own price increases.

\subsubsection*{Clear Definitions and Conditions}
\begin{itemize}
    \item \textbf{Ordinary Good:} Consumption is decreasing in its own price. This is the standard case, where $\frac{\partial x_i}{\partial p_i} < 0$.
    \item \textbf{Giffen Good:} Consumption is increasing in its own price. This is the upward-sloping portion of the demand curve, where $\frac{\partial x_i}{\partial p_i} > 0$.
\end{itemize}

\subsubsection*{Necessary Conditions for a Giffen Good}
For a good to be Giffen, two conditions must be met:
\begin{enumerate}
    \item \textbf{Inferiority:} The good must be \textbf{inferior} (demand decreases as income increases).
    \item \textbf{IE Dominance:} The Income Effect (IE), which opposes the price change, must be stronger in magnitude than the Substitution Effect (SE), which always reinforces the price change.
\end{enumerate}

\subsubsection*{Relevant Formulas and Slutsky Decomposition (LaTeX)}
The total effect of a price change on demand is expressed by the Slutsky Equation: 
$$\underbrace{\frac{\partial x_1}{\partial p_1}}_{\text{Total Effect}} = \underbrace{\frac{\partial x_1^s}{\partial p_1}}_{\text{Substitution Effect (SE)}} + \underbrace{\left(\frac{\partial x_1}{\partial m} x_1\right)}_{\text{Income Effect (IE)}}$$

\begin{center}
\begin{tabular}{llll}
\toprule
\textbf{Effect} & \textbf{Always True Sign} & \textbf{Interpretation of Sign for $p_1 \uparrow$} & \textbf{Source} \\
\midrule
SE ($\frac{\partial x_1^s}{\partial p_1}$) & Negative ($\leq 0$) & Always causes $x_1 \downarrow$ & \\
IE ($\frac{\partial x_1}{\partial m} x_1$) & Depends on $\frac{\partial x_1}{\partial m}$ & \textbf{Positive} if Inferior Good ($\frac{\partial x_1}{\partial m} < 0$) & \\
\bottomrule
\end{tabular}
\end{center}

\textbf{Giffen Condition:} When $p_1$ increases, the positive IE is larger than the negative SE: 
$$\left| \frac{\partial x_1}{\partial m} x_1 \right| > \left| \frac{\partial x_1^s}{\partial p_1} \right|$$
This results in a net increase in demand $x_1$ when $p_1$ increases.

\subsubsection*{Graphical Intuition}
A Giffen scenario occurs when the price of an essential but inferior good falls:
\begin{enumerate}
    \item \textbf{SE:} The good is now relatively cheaper; the consumer substitutes away from other goods, increasing $x_1^s$. (Movement along the original indifference curve/compensated budget line).
    \item \textbf{IE:} The consumer's real income increases significantly. Since the good is inferior, the consumer uses this increased purchasing power to buy \textit{less} of $x_1$ and more of the superior good. (Parallel shift outward).
    \item \textbf{Result:} The negative shift caused by the IE (reducing $x_1$) is so strong that it outweighs the positive SE, leading to a net reduction in $x_1$ despite the price falling (or conversely, a net increase in $x_1$ when the price rises).
\end{enumerate}

\noindent\rule{\linewidth}{0.4pt}

\subsubsection*{$\star$ Exam Focus: Shortcuts \& Pitfalls}
\begin{itemize}
    \item \textbf{Giffen = Inferior:} All Giffen goods must be inferior, but not all inferior goods are Giffen. Most inferior goods are still Ordinary because the SE dominates the IE.
    \item \textbf{IE Direction Check:} For a price increase, real income falls. If the good is inferior, the income effect must \textit{increase} demand ($\Delta x_i^n > 0$), working against the SE.
    \item \textbf{Special Case Pitfall:} For Perfect Complements, the Substitution Effect is zero. Since the total effect must equal SE + IE, and SE=0, $\Delta x_i$ is never dominated by SE/IE, meaning perfect complements cannot be Giffen.
\end{itemize}

\noindent\rule{\linewidth}{0.4pt}

\subsubsection*{Exam-Style Problem 19: Inferiority and the Total Effect}
\textbf{Problem (Conceptual Analysis):}

A consumer observes that when the price of good 1 increases, the total change in quantity demanded for good 1 ($\Delta x_1$) is positive.
\begin{enumerate}
    \item Classify good 1 as an Ordinary good or a Giffen good.
    \item What does this observation imply about the relationship between the Substitution Effect (SE) and the Income Effect (IE) for good 1?
    \item If this consumer has convex preferences, is it possible that the Substitution Effect for good 1 is zero?
\end{enumerate}

\textbf{Solution:}

\textbf{Part 1: Classification}
\begin{enumerate}
    \item \textbf{Observation:} $p_1 \uparrow$ leads to $\Delta x_1 \uparrow$.
    \item \textbf{Definition:} By definition, a good whose demand rises when its price rises is a \textbf{Giffen Good}.
\end{enumerate}

\textbf{Part 2: SE/IE Relationship}
\begin{enumerate}
    \item \textbf{SE:} When $p_1 \uparrow$, the SE must be negative, pushing demand down ($\Delta x_1^s < 0$).
    \item \textbf{IE:} Since the price increase reduces purchasing power, the IE must push demand up ($\Delta x_1^n > 0$) for the net change to be positive. Therefore, the good is inferior.
    \item \textbf{Dominance:} For the Total Effect to be positive, the magnitude of the Income Effect must be strictly greater than the magnitude of the Substitution Effect: $|\text{IE}| > |\text{SE}|$.
\end{enumerate}

\textbf{Part 3: Zero Substitution Effect}
\begin{enumerate}
    \item \textbf{Convex Preferences:} The total substitution effect is generally non-zero and negative when preferences are strictly convex (Cobb-Douglas, Quasilinear).
    \item \textbf{Zero SE Cases:} The substitution effect is only zero in special cases, such as Perfect Complements.
    \item \textbf{Conclusion:} If the SE were zero, the total effect ($\Delta x_1$) would equal the IE ($\Delta x_1^n$). Since the Total Effect is positive ($\Delta x_1 > 0$), the IE must also be positive ($\Delta x_1^n > 0$). This means that when real income falls due to $p_1 \uparrow$, demand for $x_1$ increases, classifying it as an Inferior Good. It is \textit{possible} for SE to be zero (e.g., if the consumer had perfect complement preferences), but this would contradict the Giffen condition that IE \textit{dominates} the SE, as the IE would merely equal the total change.
\end{enumerate}

\textit{(Based on standard theory, the SE is non-zero (negative) for standard convex goods. While the Giffen case is possible only if the IE dominates the SE, the case where SE=0 and IE >0 classifies the good as inferior, not Giffen (as there is no SE to dominate). However, since the observed outcome is Giffen, it is necessary that $x_1$ is inferior and $|\text{IE}| > |\text{SE}|$. If preferences were Perfect Complements, SE=0, and since we need $|\text{IE}| > 0$, the good is strictly inferior, and the total effect equals the income effect, making it Giffen. Therefore, a zero SE is \textit{\textbf{possible}} if the preferences are Perfect Complements and the good is inferior.)}

\hrule

\section{Exam Practice}
\subsection{Problem Set Practice: Optimization and Comparative Statics}

This section reviews critical problem types found in Problem Sets 1 and 2, focusing on mastering optimization techniques for special and standard preferences, and executing the Slutsky decomposition.

Strategy Check: When to Use Which Method

\begin{center}
\begin{tabular}{llll}
\toprule
Preference Type & Convexity & Optimization Method & Key Condition \\
\midrule
\textbf{Cobb-Douglas} ($x_1^c x_2^d$) & Strictly Convex & Tangency + Budget & $\frac{MU_1}{MU_2} = \frac{p_1}{p_2}$ \\
\textbf{Perfect Substitutes} ($a x_1 + b x_2$) & Convex (linear) & Corner Solution Check & $\frac{a}{p_1} \neq \frac{b}{p_2}$ (Compare MU/p ratios) \\
\textbf{Perfect Complements} ($\min\{a x_1, b x_2\}$) & Convex (kinked) & Fixed Ratio + Budget & $a x_1 = b x_2$ \\
\textbf{Quasilinear} ($v(x_1) + x_2$) & Strictly Convex & Tangency (Check feasibility) & $\frac{MU_1}{MU_2} = \frac{p_1}{p_2}$, then check if $x_i^* \geq 0$. \\
\textbf{Concave} (e.g., $x_1^2+x_2^2$) & Concave & Corner Solution Check & Compare utility at intercepts $\left(\frac{m}{p_1}, 0\right)$ vs $\left(0, \frac{m}{p_2}\right)$. \\
\bottomrule
\end{tabular}
\end{center}

\bigskip\noindent\rule{\linewidth}{0.4pt}\bigskip

\subsection*{\textbf{Exam Problem 20: Optimization with Perfect Substitutes (Disguised Form)}}

\textbf{Problem (Based on PS2 Q1(a)):}

Suppose $p_1 = 5$, $p_2 = 1$, and $m = 20$. Determine the optimal consumption bundle for the utility function $u(x_1, x_2) = 4x_1^2 + 4x_1x_2 + x_2^2$.

\textbf{Solution:}

\textbf{Step 1: Simplify the Utility Function and Identify Preference Type}
Notice that the utility function can be factored:
\[ u(x_1, x_2) = (2x_1 + x_2)^2 \]
Since $f(v) = v^2$ is a strictly monotonic transformation for $v > 0$, this function represents the same preferences as the linear function $v(x_1, x_2) = 2x_1 + 1x_2$. This is a \textbf{Perfect Substitutes} preference type, meaning the solution will likely be a corner solution. $a=2, b=1$.

\textbf{Step 2: Apply Perfect Substitutes Shortcut (Compare MU/$p$ Ratio)}
Compare the marginal utility per dollar spent for each good:
\[ \frac{MU_1}{p_1} = \frac{2}{5} = 0.4 \]
\[ \frac{MU_2}{p_2} = \frac{1}{1} = 1.0 \]

\textbf{Step 3: Determine Optimal Specialization}
Since $1.0 > 0.4$, the consumer gets more utility per dollar spent on good 2. The consumer specializes entirely in good 2.
\[ x_1^* = 0 \]

\textbf{Step 4: Calculate $x_2^*$}
\[ p_1 x_1^* + p_2 x_2^* = m \implies 5(0) + 1(x_2^*) = 20 \]
\[ \boldsymbol{x_2^* = 20} \]

\textbf{Optimal Bundle:} $(x_1^*, x_2^*) = (0, 20)$.

\bigskip\noindent\rule{\linewidth}{0.4pt}\bigskip

\subsection*{\textbf{Exam Problem 21: Optimization with Quasilinear Preferences (Corner Solution Check)}}

\textbf{Problem (Based on PS2 Q1(d)):}

Find the optimal consumption bundle for $u(x_1, x_2) = x_1 + 5 \ln(x_2)$ given prices $p_1=5$, $p_2=1$, and income $m=20$.

\textbf{Solution:}

\textbf{Step 1: Calculate Marginal Utilities and MRS}
\[ MU_1 = 1 \]
\[ MU_2 = \frac{5}{x_2} \]
\[ \text{MRS} = -\frac{MU_1}{MU_2} = -\frac{1}{5/x_2} = -\frac{x_2}{5} \]

\textbf{Step 2: Attempt Interior Solution (Tangency Condition)}
Set $|\text{MRS}|$ equal to the price ratio $p_1/p_2 = 5/1 = 5$:
\[ \frac{x_2}{5} = 5 \implies x_2^* = 25 \]

\textbf{Step 3: Check Budget Constraint Feasibility}
Substitute $x_2^*=25$ into the budget constraint $5x_1 + 1x_2 = 20$:
\[ \boldsymbol{5x_1 + 25 = 20} \]
\[ 5x_1 = -5 \implies x_1^* = -1 \]
Since $x_1^*$ must be non-negative, the interior tangency solution is \textbf{infeasible}. The optimal bundle must be a corner solution.

\textbf{Step 4: Check Corner Solutions}
\begin{enumerate}
    \item \textbf{Corner 1 (Only Good 1):} $(m/p_1, 0) = (4, 0)$.
    \[ u(4, 0) = 4 + 5 \ln(0) = -\infty \]
    (Since $\ln(0)$ is undefined, utility approaches negative infinity. This is the inferior corner).

    \item \textbf{Corner 2 (Only Good 2):} $(0, m/p_2) = (0, 20)$.
    \[ u(0, 20) = 0 + 5 \ln(20) \approx 14.98 \]
\end{enumerate}

\textbf{Conclusion:} The consumer chooses the bundle that yields the highest feasible utility. \textbf{Optimal Bundle:} $(x_1^*, x_2^*) = (0, 20)$.

\bigskip\noindent\rule{\linewidth}{0.4pt}\bigskip

\subsection*{\textbf{Exam Problem 22: Slutsky Decomposition}}

\textbf{Problem (Based on PMT Q5, different parameters):}

A consumer has demand functions $x_1^*(p_1, p_2, m) = \frac{2m}{3p_1}$ and $x_2^*(p_1, p_2, m) = \frac{m}{3p_2}$. Initially, prices are $p_1=1$, $p_2=4$, and $m=60$. The price of good 2 decreases to $p_2'=3$. Decompose the total change in demand for \textbf{good 2} into the substitution and income effects using the Slutsky method.

\textbf{Solution Strategy:} Find the initial (A), final (C), and compensated (S) bundles.

\textbf{Step 1: Initial Optimal Bundle (A)}
\[ x_2^A = \frac{60}{3(4)} = 5 \]
\[ x_1^A = \frac{2(60)}{3(1)} = 40 \]
Bundle A: $(40, 5)$.

\textbf{Step 2: Final Optimal Bundle (C)}
New prices $p'=(1, 3)$, $m=60$:
\[ x_2^C = \frac{60}{3(3)} = \frac{20}{3} \approx 6.67 \]
Total Change: $\Delta x_2 = x_2^C - x_2^A = \frac{20}{3} - 5 = \frac{20 - 15}{3} = \frac{5}{3}$

\textbf{Step 3: Calculate Compensated Income ($m^S$)}
$m^S$ is the income needed to afford $A=(40, 5)$ at new prices $p'=(1, 3)$:
\[ m^S = p_1' x_1^A + p_2' x_2^A = 1(40) + 3(5) = 40 + 15 = 55 \]

\textbf{Step 4: Calculate Compensated Demand ($x_2^S$)}
Use $p'=(1, 3)$ and $m^S=55$:
\[ x_2^S = \frac{55}{3(3)} = \frac{55}{9} \approx 6.11 \]

\textbf{Step 5: Calculate Substitution Effect (SE)}
\[\Delta x_2^s = x_2^S - x_2^A = \frac{55}{9} - 5 = \frac{55}{9} - \frac{45}{9} = \frac{10}{9}\]

\textbf{Step 6: Calculate Income Effect (IE)}
\[\Delta x_2^n = x_2^C - x_2^S = \frac{20}{3} - \frac{55}{9} = \frac{60}{9} - \frac{55}{9} = \frac{5}{9}\]

\textbf{Step 7: Verification and Classification}
Total Effect: $\Delta x_2 = \frac{10}{9} (\text{SE}) + \frac{5}{9} (\text{IE}) = \frac{15}{9} = \frac{5}{3}$. (Matches Step 2).
Since $\Delta x_2^n > 0$ when purchasing power increases ($p_2 \downarrow$ makes $m^S < m$), good 2 is a \textbf{Normal Good}.

\begin{center}
\begin{tabular}{lll}
\toprule
Effect & Value (Fraction) & Value (Decimal) \\
\midrule
\textbf{Total Change ($\Delta x_2$)} & $5/3$ & $\approx 1.67$ \\
\textbf{Substitution Effect ($\Delta x_2^s$)} & $10/9$ & $\approx 1.11$ \\
\textbf{Income Effect ($\Delta x_2^n$)} & $5/9$ & $\approx 0.56$ \\
\bottomrule
\end{tabular}
\end{center}

\bigskip\noindent\rule{\linewidth}{0.4pt}

\section{Exam Practice (Continued)}
\subsection{Practice Midterm Review}

The Practice Midterm Exam provides the best guide to the structure, style, and required rigor for ECON 340 Exam 1. This section summarizes the core concepts tested in the practice midterm, ensuring focus on calculations and formal arguments.

\textbf{Exam Structure Focus (6 Long-Form Questions)}

The practice midterm covers the following major topics, emphasizing quantitative analysis:

\begin{enumerate}
\item \textbf{Preferences and Convexity (Conceptual/Proof):} Testing definitions of rationality (completeness, transitivity) and strict convexity using consumption bundles.
\item \textbf{Utility and MRS Analysis (Calculus/Interpretation):} Calculating MRS and mathematically proving diminishing MRS (convexity).
\item \textbf{Consumer Optimization (Interior/Corner Solutions):} Solving utility maximization for standard (Cobb-Douglas) and special (Quasilinear) preferences.
\item \textbf{Revealed Preference (Consistency Test):} Applying WARP to determine if observed choices are rational.
\item \textbf{Comparative Statics (Slutsky Decomposition):} Calculating the total change in demand and splitting it into Substitution and Income effects.
\item \textbf{Non-Convex Optimization (Demand Function Derivation):} Deriving demand functions for concave preferences, necessitating a corner solution comparison.
\end{enumerate}

\vspace{1em}\hrule\vspace{1em}

\subsection*{Exam Problem 23: Testing Convexity using Bundle Comparison (PMT Q1)}

\textbf{Problem (Adapted from PMT Q1):}

A consumer is indifferent between $X=(7, 2)$ and $Y=(3, 8)$. Preferences are strictly monotonic and rational. The consumer strictly prefers $A=(2, 7)$ over $B=(5, 6)$. Is the preference convex? Explain.

\textbf{Solution:}

\textbf{Strategy:} Convexity requires that mixtures are preferred to extremes. We check the utility of the 50-50 mixture of the two indifferent bundles $X$ and $Y$.

\begin{enumerate}
\item \textbf{Calculate the Mixture Bundle ($Z$):} \[ Z = \frac{1}{2}(7, 2) + \frac{1}{2}(3, 8) = (3.5 + 1.5, 1 + 4) = (5, 5) \] For convexity, we must have $Z \succeq X$ and $Z \succeq Y$. (A mixture is weakly preferred to the extremes).
\item \textbf{Use Given Preferences and Monotonicity/Transitivity:} We are given $X \sim Y$. We must compare $Z=(5, 5)$ to $X$ and $Y$.
\begin{itemize}
\item We compare $B=(5, 6)$ and $Z=(5, 5)$. Since $B$ has more of good 2 and the same amount of good 1, by strict monotonicity: \[ B \succ Z \]
\item We are given the strict preference: $A=(2, 7)$ over $B=(5, 6)$: \[ A \succ B \]
\item We compare $Y=(3, 8)$ and $A=(2, 7)$. By strict monotonicity: \[ Y \succ A \]
\end{itemize}
\item \textbf{Apply Transitivity (Check Consistency):} We established the following chain of strict preference: \[ Y \succ A \succ B \succ Z \] By transitivity, $Y \succ Z$.
\item \textbf{Conclusion:} Since $Y$ is strictly preferred to the mixture $Z$ ($Y \succ Z$), the mixture is inferior to the extreme. This violates the definition of convex preferences. \textbf{Answer:} The preference is \textbf{not convex}.
\end{enumerate}

\vspace{1em}\hrule\vspace{1em}

\subsection*{Exam Problem 24: Quasilinear Optimization (PMT Q3b)}

\textbf{Problem (Adapted from PMT Q3b):}

Suppose $p_1=1$, $p_2=2$, and $m=10$. Find the optimal consumption bundle for $u(x_1, x_2) = 8 \ln(x_1) + x_2$.

\textbf{Solution:}

\textbf{Strategy:} This is a quasilinear utility function, which is strictly convex. We attempt the interior solution first.

\textbf{Step 1: Calculate Marginal Utilities and MRS}
\[ MU_1 = \frac{8}{x_1}, \quad MU_2 = 1 \]
\[ \text{MRS} = -\frac{MU_1}{MU_2} = -\frac{8}{x_1} \]

\textbf{Step 2: Apply the Tangency Condition}
Set $|\text{MRS}|$ equal to the price ratio $p_1/p_2 = 1/2$:
\[ \frac{8}{x_1} = \frac{1}{2} \implies x_1^* = 16 \]

\textbf{Step 3: Check Budget Constraint Feasibility}
Substitute $x_1^*=16$ into the budget equation $1x_1 + 2x_2 = 10$:
\[ 16 + 2x_2 = 10 \]
\[ 2x_2 = 10 - 16 = -6 \]
\[ x_2^* = -3 \]
Since consumption must be non-negative ($x_2 \geq 0$), the interior solution is \textbf{infeasible}.

\textbf{Step 4: Corner Solution}
The optimum must be a corner solution, specifically $x_2^*=0$ or $x_1^* =0$.

\begin{enumerate}
\item \textbf{Corner 1 ($x_2=0$):} $(m/p_1, 0) = (10, 0)$.
\[ u(10, 0) = 8 \ln(10) + 0 \approx 18.42 \]
\item \textbf{Corner 2 ($x_1=0$):} $(0, m/p_2) = (0, 5)$.
\[ u(0, 5) = 8 \ln(0) + 5 = -\infty \]
(Since $\ln(0)$ is undefined/approaches $-\infty$).
\end{enumerate}

\textbf{Optimal Bundle:} The highest feasible utility is achieved at $(10, 0)$.

\vspace{1em}\hrule\vspace{1em}

\subsection*{Exam Problem 25: Concave Preferences Demand (PMT Q6)}

\textbf{Problem (Adapted from PMT Q6):}

The consumer’s utility function is given by $u(x_1, x_2) = 4x_1^2 + x_2^2$.
\begin{enumerate}
\item Derive the demand function $x_1^*(p_1, p_2, m)$ and $x_2^*(p_1, p_2, m)$.
\item Suppose $p_2 = 3$ and $m = 18$. Draw the demand curve for good 1 (price $p_1$ on the vertical axis).
\end{enumerate}

\textbf{Solution:}

\textbf{Part 1: Demand Function Derivation}

\textbf{Strategy:} This is a strictly concave preference (see Section 2c, Exam Problem 6), meaning the optimum is always a corner solution. We compare utility at the intercepts.

\begin{enumerate}
\item \textbf{Candidate 1 (Only Good 2):} $x^A = (0, m/p_2)$.
\[ u(x^A) = 4(0)^2 + \left(\frac{m}{p_2}\right)^2 = \frac{m^2}{p_2^2} \]
\item \textbf{Candidate 2 (Only Good 1):} $x^B = (m/p_1, 0)$.
\[ u(x^B) = 4\left(\frac{m}{p_1}\right)^2 + 0^2 = \frac{4m^2}{p_1^2} \]
\item \textbf{Determine the Switching Condition:} The consumer chooses Candidate 2 if $u(x^B) > u(x^A)$:
\[ \frac{4m^2}{p_1^2} > \frac{m^2}{p_2^2} \]
Since $m^2 > 0$, we simplify:
\[ 4p_2^2 > p_1^2 \implies \sqrt{4p_2^2} > \sqrt{p_1^2} \implies 2p_2 > p_1 \]
\item \textbf{Demand Functions (LaTeX):}
\[ x_1^*(p_1, p_2, m) = \begin{cases} m/p_1 & \text{if } p_1 < 2p_2 \\ 0 & \text{if } p_1 > 2p_2 \end{cases} \]
\[ x_2^*(p_1, p_2, m) = \begin{cases} 0 & \text{if } p_1 < 2p_2 \\ m/p_2 & \text{if } p_1 > 2p_2 \end{cases} \]
(If $p_1 = 2p_2$, the consumer is indifferent between the two corner solutions).
\end{enumerate}

\textbf{Part 2: Drawing the Demand Curve for $x_1$}

Given $p_2 = 3$ and $m = 18$. The switching price is $p_1 = 2p_2 = 2(3) = 6$.

\begin{enumerate}
\item \textbf{Case 1: $p_1 < 6$} (Good 1 is relatively cheap): The consumer buys only good 1: $x_1 = m/p_1 = 18/p_1$.
\begin{itemize}
\item If $p_1 \to 0$, $x_1 \to \infty$.
\item If $p_1 = 6$, $x_1 = 18/6 = 3$.
\end{itemize}
\item \textbf{Case 2: $p_1 > 6$} (Good 2 is relatively cheap): The consumer buys only good 2: $x_1 = 0$.
\end{enumerate}

\textbf{Shape:} The demand curve is a standard downward-sloping demand curve (a segment of a rectangular hyperbola $x_1 = 18/p_1$), which drops vertically to zero at $x_1=0$ when the price exceeds the switching price of $p_1=6$ [436, Figure 2].

\begin{table}[h!]
\centering
\begin{tabular}{ll}
\toprule
\textbf{Price Range} & \textbf{Demand $x_1$} \\
\midrule
$0 \leq p_1 < 6$ & $18/p_1$ \\
$p_1 > 6$ & $0$ \\
\bottomrule
\end{tabular}
\end{table}

\hrulefill

\section*{8. Exam Practice (Continued)}

\subsection*{(c) Practice Questions Packet}

This section targets essential concepts of budget constraints, preference classification, and consumer demand derived from the practice questions provided, ensuring rapid review of key calculations.

\subsubsection*{Exam Problem 26: Budget Constraint Fundamentals}

\textbf{Problem (Based on Practice Q1 \& Q2):}
\begin{enumerate}
    \item Your budget line equation is given by $10x_1 + 4x_2 = m$. In order to obtain 1 more unit of good 1, how many units of good 2 must you give up?
    \item Suppose bundles $A=(3, 9)$ and $B=(1, 10)$ lie on your budget line. If you spent your entire income on good 1, how many units of good 1 could you buy?
\end{enumerate}

\textbf{Solution:}

\textbf{Part 1: Opportunity Cost}
\begin{enumerate}
    \item \textbf{Identify Prices:} The budget line $p_1 x_1 + p_2 x_2 = m$ implies $p_1=10$ and $p_2=4$.
    \item \textbf{Calculate Opportunity Cost:} The rate at which the market allows trade is the negative slope of the budget line, $|-\frac{p_1}{p_2}|$.
    \[\text{Opportunity Cost of } x_1 = \frac{p_1}{p_2} = \frac{10}{4} = 2.5\]
    \item \textbf{Interpretation:} To get 1 more unit of good 1, you must give up \textbf{2.5} units of good 2.
\end{enumerate}

\textbf{Part 2: Calculating Intercept ($m/p_1$)}
\begin{enumerate}
    \item \textbf{Find the Slope:} Calculate the slope using the two points $A=(3, 9)$ and $B=(1, 10)$:
    \[\text{Slope} = \frac{\Delta x_2}{\Delta x_1} = \frac{10 - 9}{1 - 3} = \frac{1}{-2} = -0.5\]
    \item \textbf{Determine Price Ratio:} Since $\text{Slope} = -p_1/p_2$, we have $p_1/p_2 = 0.5$. Thus, $p_2 = 2p_1$.
    \item \textbf{Find Income ($m$):} Use the budget equation $p_1 x_1 + p_2 x_2 = m$ and bundle $A=(3, 9)$:
    \[ m = p_1(3) + p_2(9) \]
    Substitute $p_2 = 2p_1$:
    \[ m = 3p_1 + 2p_1(9) = 3p_1 + 18p_1 = 21p_1 \]
    \item \textbf{Calculate Horizontal Intercept:} The maximum amount of good 1 is $m/p_1$:
    \[\text{Horizontal Intercept} = \frac{m}{p_1} = \frac{21 p_1}{p_1} = 21\]
    (Answer: \textbf{21} units).
\end{enumerate}

\bigskip\hrule\bigskip

\subsubsection*{Exam Problem 27: Identifying Special Preference Forms}

\textbf{Problem (Based on Practice Q6 \& Q8):}
\begin{enumerate}
    \item Which utility function best represents preferences where goods are consumed together in a fixed 2:1 ratio (2 units of $x_1$ for every 1 unit of $x_2$)?
    \begin{enumerate}
        \renewcommand{\theenumii}{\alph{enumii}}
        \renewcommand{\labelenumii}{(\theenumii)}
        \item $u(x_1, x_2) = 2x_1 + x_2$.
        \item $u(x_1, x_2) = \min\{2x_1, x_2\}$.
        \item $u(x_1, x_2) = x_1 + 2x_2$.
        \item $u(x_1, x_2) = \min\{x_1, 2x_2\}$.
    \end{enumerate}
    \item Suppose a consumer's utility is given by $u(x_1, x_2) = (x_1 + x_2)/2$. What property must be true about the consumer’s indifference curves?
\end{enumerate}

\textbf{Solution:}

\textbf{Part 1: Fixed Ratio (Perfect Complements)}
\begin{enumerate}
    \item \textbf{Required Ratio:} The consumer needs $2x_1 = 1x_2$ (or $x_1/x_2 = 1/2$).
    \item \textbf{Perfect Complements Form:} $u(x_1, x_2) = \min\{a x_1, b x_2\}$. Optimality occurs at the kink where $a x_1 = b x_2$.
    \item \textbf{Match to Required Ratio:} If we choose $a=1$ and $b=2$, the kink condition is $1x_1 = 2x_2$, which means $x_1/x_2 = 2/1$. (Incorrect).
    \item If we choose $a=2$ and $b=1$, the kink condition is $2x_1 = 1x_2$, which means $x_1/x_2 = 1/2$. (Correct).
    \item \textbf{Answer:} (b) $u(x_1, x_2) = \min\{2x_1, x_2\}$.
\end{enumerate}

\textbf{Part 2: Linear Utility}
\begin{enumerate}
    \item \textbf{Analyze Function:} $u(x_1, x_2) = \frac{x_1 + x_2}{2}$. This is a Perfect Substitutes form, equivalent to $v(x_1, x_2) = x_1 + x_2$, since division by 2 is a positive monotonic transformation.
    \item \textbf{Indifference Curve:} If $u(x_1, x_2) = k$, then $x_1 + x_2 = 2k$, or $x_2 = 2k - x_1$.
    \item \textbf{Conclusion:} The equation is that of a straight line with a constant slope of $-1$. The consumer has \textbf{linear indifference curves}.
\end{enumerate}

\bigskip\hrule\bigskip

\subsubsection*{Exam Problem 28: Comparative Statics and Good Classification (Calculus)}

\textbf{Problem (Based on Practice Q24, Calculus Application):}

A consumer has the demand function for good 1 given by $x_1(p_1, p_2, m) = \frac{m}{p_1 + p_2}$.
\begin{enumerate}
    \item Is good 1 a Normal Good or an Inferior Good?
    \item Are goods 1 and 2 substitutes or complements?
    \item Calculate the own-price change (Total Effect) $\frac{\partial x_1}{\partial p_1}$.
\end{enumerate}

\textbf{Solution:}

\textbf{Part 1: Normal or Inferior Good? (Effect of Income $m$)}
\begin{enumerate}
    \item \textbf{Calculate Income Derivative:}
    \[ \frac{\partial x_1}{\partial m} = \frac{\partial}{\partial m} \left( \frac{m}{p_1 + p_2} \right) = \frac{1}{p_1 + p_2} \]
    \item \textbf{Determine Sign:} Since prices are positive, $\frac{\partial x_1}{\partial m} > 0$.
    \item \textbf{Conclusion:} Good 1 is a \textbf{Normal Good} (demand increases with income).
\end{enumerate}

\textbf{Part 2: Substitutes or Complements? (Effect of Cross-Price $p_2$)}
\begin{enumerate}
    \item \textbf{Calculate Cross-Price Derivative:}
    \[ \frac{\partial x_1}{\partial p_2} = \frac{\partial}{\partial p_2} \left( m (p_1 + p_2)^{-1} \right) = m \cdot (-1) (p_1 + p_2)^{-2} \cdot 1 \]
    \[ \frac{\partial x_1}{\partial p_2} = -\frac{m}{(p_1 + p_2)^2} \]
    \item \textbf{Determine Sign:} Since $m$ and prices are positive, $\frac{\partial x_1}{\partial p_2} < 0$.
    \item \textbf{Conclusion:} An increase in the price of good 2 decreases the demand for good 1. Goods 1 and 2 are \textbf{Complements} (Gross Complements).
\end{enumerate}

\textbf{Part 3: Total Effect of Own Price Change ($p_1$)}
\begin{enumerate}
    \item \textbf{Calculate Own-Price Derivative:}
    \[ \frac{\partial x_1}{\partial p_1} = \frac{\partial}{\partial p_1} \left( m (p_1 + p_2)^{-1} \right) = m \cdot (-1) (p_1 + p_2)^{-2} \cdot 1 \]
    \[ \frac{\partial x_1}{\partial p_1} = -\frac{m}{(p_1 + p_2)^2} \]
    \item \textbf{Interpretation:} Since $\frac{\partial x_1}{\partial p_1} < 0$, the good is an Ordinary Good, satisfying the Law of Demand.
\end{enumerate}

\subsubsection*{$\star$ Exam Interpretation Pitfalls}
\begin{itemize}
    \item \textbf{Complement Definition:} A negative cross-price derivative ($\frac{\partial x_1}{\partial p_2} < 0$) indicates that goods are (gross) complements. If the price of $p_2$ goes up, the consumer buys less of $x_1$.
    \item \textbf{Normal Good Check:} Since $x_1$ is a Normal Good, the Total Effect must be negative (Part 3), because the SE and IE both reduce demand when $p_1$ increases.
\end{itemize}

\hrulefill


\end{document}
